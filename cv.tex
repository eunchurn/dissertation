\chapter*{Curriculum Vit\ae}
\addcontentsline{toc}{chapter}{Curriculum Vit\ae}
%\setheader{Curriculum Vit\ae}

%% Print the full name of the author.
% \makeatletter
% \authors{Eunchurn Park}
% \makeatother

\section*{Eunchurn Park}
\noindent
\begin{tabular}{p{4\parindent}l}
    1980.07.05 & Born in Iksan, Korea.
\end{tabular}
\subsection*{Contact Information}

% NOTE: Mind where the & separators and \\ breaks are in the following
%       table.
%
% ALSO: \rcollength is the width of the right column of the table
%       (adjust it to your liking; default is 1.85in).
%
%\newlength{\rcollength}\setlength{\rcollength}{1.4in}%
%
\begin{tabularx}{\textwidth}{@{}XX@{}}
%\href{http://www.cse.osu.edu/}%
%     {Department of Computer Science and Engineering} & \\
%\href{http://www.osu.edu/}{The Ohio State University}
1103, 619-3 Gyeongin-ro    & +82-10-4499-6420 \\
Guro-gu, Seoul, Korea, 08210     & eunchurn.park@gmail.com\\
\end{tabularx}




\subsection*{Education}
\subsubsection*{\textit{History}}
\begin{tabular}{p{4\parindent}l}
    1996--1999 & Bugil High School  \\
    & 69 Dandae Ro, Dongnam-gu, Cheonan City\\
    1999--2005 & Bachelor of science, dept. of architectural engineering\\
    & School of Architecture, Dankook University, Republic of Korea \\
    \\
    2005--2007 & Master of science, dept. of architectural engineering\\
    & School of Architecture, Dankook University, Republic of Korea  \\
    \\
    2007--2017 & Ph.D. \\
    & School of Architecture, Dankook University, Republic of Korea  \\
    %% The width of the second column is the width of the page, minus the width
    %% of the first column (4\parindent) minus four times the separation between
    %% the start of the column and its contents.
    % \begin{minipage}{\textwidth-4\parindent-4\tabcolsep}
    %     %% We divide the minipage 20/80.
    %     \begin{tabular}{@{}p{0.2\linewidth}@{}p{0.8\linewidth-\tabcolsep}}
    %         \textit{Thesis:} & A \\
    %         \textit{Promotor:} & Prof. dr. A. Kleiner
    %     \end{tabular}
    % \end{minipage}
\end{tabular}

\subsubsection*{\textit{Qualification}}

\href{http://www.dankook.ac.kr/}{\textbf{Dankook University}},
\begin{itemize}
\item[] Ph.D.,
        \href{http://cms.dankook.ac.kr/web/archi}
             {Architectural Engineering, School of Architecture},
             Aug. 2017
        \begin{itemize}
        \item Topic: \emph{Hybrid Testing Method and Excitation System Design for Seismic and Wind-resistance Performance Evaluation of Building Structures with Nonlinear Dampers}
        \item Advisors:
              \href{http://cms.dankook.ac.kr/web/archi/-16?p_p_id=DeptInfo_WAR_empInfoportlet&p_p_lifecycle=0&p_p_state=normal&p_p_mode=view&p_p_col_id=column-2&p_p_col_count=1&_DeptInfo_WAR_empInfoportlet_empId=2zEyEnhbhLlys2HRljBFWg%3D%3D&_DeptInfo_WAR_empInfoportlet_action=view_message}
                   {Kyung-Won Min, Ph.D}
        \end{itemize}

\item[] M.S.,
        \href{http://cms.dankook.ac.kr/web/archi}
             {Architectural Engineering, School of Architecture},
             Feb. 2007
        \begin{itemize}
        \item Topic: \emph{Design of Excitation Systems for Simulating Dynamic Loads and Real-Time Hybrid Test Method of Building Structures}
        \item Advisor:
              \href{http://cms.dankook.ac.kr/web/archi/-16?p_p_id=DeptInfo_WAR_empInfoportlet&p_p_lifecycle=0&p_p_state=normal&p_p_mode=view&p_p_col_id=column-2&p_p_col_count=1&_DeptInfo_WAR_empInfoportlet_empId=2zEyEnhbhLlys2HRljBFWg%3D%3D&_DeptInfo_WAR_empInfoportlet_action=view_message}
                   {Kyung-Won Min, Ph.D}
        \end{itemize}
\item[] B.S.,
        \href{http://cms.dankook.ac.kr/web/archi}
             {Architectural Engineering, School of Architecture},
             Feb. 2005
\end{itemize}

\subsection*{Work Experience}

\begin{tabularx}{\textwidth}{@{}p{5\parindent}X@{}}
2009.04 - 2015.05 & \href{http://www.kmbest.co.kr/}{한국유지관리(주)} 
부설기업연구소 : R\&D \textgreater{} U-biz, DAQ developemtn, Measurement of civil infra-structure, Structural health monitoring\\
2015.08 - 현재 & \href{http://www.shtpi.co.kr/}{(주)승화기술정책연구소}
 기술연구소 : R\&D \textgreater{} Computer vision/IoT, M2M Platform development/Full stack engineering/Deep learning/Railway projects\\
\end{tabularx}


\subsection*{Awards}

\begin{tabularx}{\textwidth}{p{4\parindent}X}
    2007 & Excellence Prize, The 3rd Excellent Graduation Theses of \href{http://www.aik.or.kr}{The Architectural Institute of Korea} \\
    & 석사부문 우수상, 제3회 대한건축학회 우수졸업논문전 \\
    \\
    2007 & 2007 Beomjeong Academic Paper Prize of \href{http://cms.dankook.ac.kr/web/grad}{Dankook University Graduate School} \\
    & 2007 범정학술논문상 \\
    \\
    2008 & 2008 Beomjeong Academic Paper Prize of \href{http://cms.dankook.ac.kr/web/grad}{Dankook University Graduate School} \\
    & 2008 범정학술논문상 \\
    \\
    2012 & The 19th Science and Technology Best Paper Award \\
    & 제19회 과학기술우수논문상 (한국소음진동공학회)
    \\
\end{tabularx}

\subsection*{Publications}
\subsubsection*{Papers in international journal (SCI)}
\begin{itemize}
\item[]
  \textbf{Eunchurn Park}, Kyung-Won Min, Sung-Kyung Lee, Sang-Hyun Lee,
  Heon-Jae Lee, Seok-Jun Moon, Hyung-Jo Jung, ``Real-time Hybrid Test on
  a Semi-actively Controlled Building Structure Equipped with Full-scale
  MR Dampers'', \emph{Journal of Intelligent Material Systems and
  Structures} 12/2010; 21(18):1831-1850. DOI:10.1177/1045389X10390253
\item[]
  Heon-Jae Lee, Hyung-Jo Jung, Seok-Jun Moon, Sung-Kyung Lee,
  \textbf{Eun-Churn Park}, Kyung-Won Min, ``Experimental Investigation
  of MR Damper-based Semiactive Control Algorithms for Full-scale
  Five-story Steel Frame Building'', \emph{Journal of Intelligent
  Material Systems and Structures} 06/2010; 21(10):1025-1037.
  DOI:10.1177/1045389X10374162 
\item[]
  Jae-Sung Heo, Sung-Kyung Lee, \textbf{Eunchurn Park}, Sang-Hyun Lee,
  Kyung-Won Min, Hongjin Kim, Jiseong Jo, Bong-Ho Cho, ``Performance
  test of a tuned liquid mass damper for reducing bidirectional
  responses of building structures'', \emph{The Structural Design of
  Tall and Special Buildings} 11/2009; 18(7):789 - 805.
  DOI:10.1002/tal.486
\item[]
  \textbf{Eun-Churn Park}, Sang-Hyun Lee, Kyung-Won Min, Lan Chung,
  Sung-Kyung Lee, Seung-Ho Cho, Eunjong Yu, Kyung-Soo Kang, ``Design of
  an actuator for simulating wind-induced response of a building
  structure'', \emph{SMART STRUCTURES AND SYSTEMS} 01/2008; 4(1).
  DOI:10.12989/sss.2008.4.1.085
\item[]
  Sung-Kyung Lee, \textbf{Eun Churn Park}, Kyung-Won Min, Ji-Hun Park,
  ``Real-time substructuring technique for the shaking table test of
  upper substructures'', \emph{Engineering Structures} 09/2007;
  29(9):2219-2232. DOI:10.1016/j.engstruct.2006.11.013
\item[]
  Sung-Kyung Lee, \textbf{Eun Churn Park}, Kyung-Won Min, Sang-Hyun Lee,
  Ji-Hun Park, ``Experimental implementation of a building structure
  with a tuned liquid column damper based on the real-time hybrid
  testing method'', \emph{Journal of Mechanical Science and Technology}
  06/2007; 21(6):885-890. DOI:10.1007/BF03027063 
\item[]
  Sung-Kyung Lee, \textbf{Eun Churn Park}, Kyung-Won Min, Sang-Hyun Lee,
  Lan Chung, Ji-Hun Park, " Real-time hybrid shaking table testing
  method for the performance evaluation of a tuned liquid damper
  controlling seismic response of building structures``, \emph{Journal
  of Sound and Vibration} 05/2007; DOI:10.1016/j.jsv.2006.12.006
\end{itemize}

\subsubsection*{Papers in Korean journal (KCI)}
\begin{itemize}
\item[]
  민경원, 이성경, \textbf{박은천}, ``진동대실험에 의한
  동조액체기둥감쇠기의 동적특성'', \emph{한국소음진동공학회논문집
  (Transactions of the Korean society for noise and vibration
  engineering)} v.19 no.6 = no.147, pp.620 - 627, 2009, 1598-2785
\item[]
  윤경조, \textbf{박은천}, 이헌재, 문석준, 민경원, 정형조, 이상현, `준능동 MR감쇠기가 설치된 실물크기 구조물의 분산제어 알고리즘 성능평가'', \emph{한국소음진동공학회논문집
  (Transactions of the Korean society for noise and vibration
  engineering)} v.18 no.2, pp.255 - 262, 2008,
\item[]
  \textbf{박은천}, 이성경, 이헌재, 문석준, 정형조, 민경원, ``대형
  MR감쇠기가 설치된 건축구조물의 실시간 하이브리드 실험 및 준능동
  알고리즘 적용'', \emph{한국전산구조공학회논문집 (Journal of the
  computational structural engineering institute of Korea)} v.21 no.5,
  pp.465 - 474, 2008, 1229-3059
\item[]
  이성경, 민경원, \textbf{박은천}, ``TMD와 TLCD를 이용한 2방향 감쇠기의
  동적특성'', \emph{한국전산구조공학회논문집 (Journal of the
  computational structural engineering institute of Korea)} v.21 no.6,
  pp.589 - 596, 2008, 1229-3059
\item[]
  허재성, 이성경, \textbf{박은천}, 이상현, 김홍진, 조지성, 조봉호,
  ``실시간 하이브리드 진동대 실험법에 의한 양방향 TLMD의 진동제어
  성능평가'', \emph{한국소음진동공학회논문집 (Transactions of the Korean
  society for noise and vibration engineering)} v.18 no.5 = no.134,
  pp.485 - 495, 2008, 1598-2785
\item[]
  허재성, \textbf{박은천}, 이상현, 이성경, 김홍진, 조봉호, 조지성,
  김동영, 민경원, ``건축구조물의 2방향 진동제어를 위한
  동조액체질량감쇠기'', \emph{한국소음진동공학회논문집 (Transactions of
  the Korean society for noise and vibration engineering)} v.18 no.3 =
  no.132, pp.345 - 355, 2008, 1598-2785
\item[]
  허재성, \textbf{박은천}, 이성경, 이상현, 김홍진, 조지성, 조봉호,
  주석준, 민경원, ``실물크기 구조물에 설치된 동조액체질량감쇠기의
  성능실험'', \emph{한국전산구조공학회논문집 (Journal of the
  computational structural engineering institute of Korea)} v.21 no.2,
  pp.161 - 168, 2008, 1229-3059
\item[]
  윤경조, \textbf{박은천}, 이헌재, 문석준, 민경원, 정형조, 이상현,
  ``준능동 MR감쇠기가 설치된 실물크기 구조물의 분산제어 알고리즘
  성능평가'', \emph{한국소음진동공학회논문집 (Transactions of the Korean
  society for noise and vibration engineering)} v.18 no.2 = no.131,
  pp.255 - 262, 2008, 1598-2785
\item[]
  \textbf{박은천}, 이성경, 윤경조, 정희산, 이헌재, 최강민, 문석준,
  정형조, 민경원, ``실시간 하이브리드 실험법을 이용한 대형 MR감쇠기의
  제진 성능평가'', \emph{한국소음진동공학회논문집 (Transactions of the
  Korean society for noise and vibration engineering)} v.18 no.1 =
  no.130, pp.131 - 138, 2008, 1598-2785
\item[]
  \textbf{박은천}, 민경원, 정란, 강경수, 이상현, ``건축구조물의 풍하중
  구현 및 풍특성 평가를 위한 가진시스템 설계'',
  \emph{한국전산구조공학회논문집 (Journal of the computational
  structural engineering institute of Korea)} v.20 no.6, pp.769 - 778,
  2007, 1229-3059
\item[]
  이성경, 민경원, \textbf{박은천}, ``진동대를 이용한 구조물의 하이브리드
  실험'', \emph{대한건축학회논문집大韓建築學會論文集 (Journal of the
  architectural institute of Korea : Structure \& construction) /
  構造系} v.22 no.5 = no.211, pp.57 - 63, 2006, 1226-9107
\item[]
  이상현, \textbf{박은천}, 윤경조, 이성경, 유은종, 민경원, 정란, 민정기,
  김영찬, ``실물 크기 구조물의 강제진동실험 및 지진응답 모사를 위한
  HMD제어기 설계'', \emph{한국지진공학회논문집 (Journal of the
  earthquake engineering society of Korea)} v.10 no.6 = no.52, pp.103 -
  114, 2006, 1226-525x
\item[]
  이성경, \textbf{박은천}, 이상현, 정란, 우성식, 민경원, ``실시간
  하이브리드 진동대 실험법을 이용한 TLD 제어성능의 실험적 검증'',
  \emph{한국전산구조공학회논문집 (Journal of the computational
  structural engineering institute of Korea)} v.19 no.4 = no.74, pp.419
  - 427, 2006, 1229-3059
\end{itemize}

\subsubsection*{Papers in international proceedings}
\begin{itemize}
\item[]
  Byungkyu Moon, Hosin David Lee, \textbf{Eunchurn Park}, Sungbaek Park, ``Preliminary Development of Automated Crack Measurement System for Concrete Slab Ballast of High-speed Railroad Tracks in Korea'', \emph{The 2017 International Conference on Maintenance and Rehabilitation of Constructed Infrastructure Facilities (2017 MAIREINFRA)}; 7/2017
\item[]
  Mi-Yun Park, \textbf{Eunchurn Park}, Byung-Gyu Moon, Wan-Sun Park, Se-Gon Kwon, Won-Yong Koo, ``Development of an IoT-based Sensor Network System for Initial Emergency Evacuating Response System in Urban Underground Space'', \emph{1st Asian Conference on Railway Infrastructure and Transportation (ART 2016)}; 10/2016
\item[]
  \textbf{Eunchurn Park}, Yu-Seung Kim, Joong-Yub Lee, Jun-Sung Choi,
  Yeon-Back Jung, Won-Kyun Seok, Kwang-Soo Jung, Soon-Jeon Park, Joo-Ho
  Lee, ``Study on Vertical Alignment Maintenance Technique using GNSS in
  Skyscraper'', \emph{2009 International Symposium on GPS/GNSS}; 11/2009
\item[]
  \textbf{Eunchurn Park}, Sung-Kyung Lee, Heon-Jae Lee, Seok-Joon Moon,
  Hyung-Jo Jung, Byoung-Wook Moon, Kyung-Won Min, ``Real-Time Hybrid
  Testing of a Steel-Structure Equipped With Large-Scale
  Magneto-Rheological Dampers Applying Semi-Active Control Algorithms'',
  \emph{ASME 2008 Conference on Smart Materials, Adaptive Structures and
  Intelligent Systems}; 01/2008
\item[]
  \textbf{Eunchurn Park}, Sang-Hyun Lee, Sung-Kyung Lee, Hee-San Chung,
  Kyung-Won Min, ``System Identification and Pseudo-Earthquake
  Excitation of a Real-Scaled 5 Story Steel Frame Structure'',
  \emph{ASME 2008 Conference on Smart Materials, Adaptive Structures and
  Intelligent Systems}; 01/2008
\item[]
  S. H. Lee, \textbf{E. C. Park}, K. J. Youn, K. W. Min, L. Chung, S. B.
  Choi, K. G. Sung, H. G. Lee, ``RESPONSE CONTROL OF A REAL-SCALED
  FIVE-STORY STRUCTURE USING MAGNETO-RHEOLOGICAL DAMPER'',
  \emph{Electrorheological Fluids and Magnetorheological Suspensions -
  10th International Conference on ERMR 2006}; 01/2007
\end{itemize}

\subsubsection*{Papers in Korean conference proceedings}
\begin{itemize}

\item[]
  \textbf{박은천}, 강형구, 최준성, 김은성, 김만철, ``고속열차의 주행
  동적성능 평가시스템 개발'', \emph{한국철도학회 2011년도 정기총회 및
  추계학술대회 논문집} Oct. 20, pp.3226-3236 (2011)
\item[]
  \textbf{박은천}, 최준성, ``전단파 속도계측에 의한 구조물 강도추정
  실용화 연구'', \emph{한국방재학회 2011년도 정기 학술발표대회} Feb. 24,
  pp.162-162. (2011)
\item[]
  김성용, textbf{박은천}, ``차량의 가속도 진동계측 및 차량안정성
  프로그램 개발에 대한 연구'', \emph{한국철도학회 2010년도 춘계학술대회
  논문집} June 10, pp.527-532. (2010)
\item[]
  김성용, \textbf{박은천}, ``차량의 동적 특성 분석을 통한 궤도틀림
  식별'', \emph{한국철도학회 2010년도 춘계학술대회 논문집} 2010 June 10,
  pp.2095-2106. (2010)
\item[]
  민경원, 이성경, \textbf{박은천}, ``1 개의 TLD 를 이용한 건물의 양방향
  진동제어'', \emph{한국전산구조공학회 2009년도 정기 학술대회} 2009 Apr.
  16, pp.119 - 124, 2009
\item[]
  허재성, 이성경, \textbf{박은천}, 이상현, 김홍진, 조지성, 조봉호,
  민경원, ``건축구조물의 2방향 진동제어를 위한 TLMD 제어성능평가'',
  \emph{한국소음진동공학회 2008년도 춘계학술대회논문집} 2008 Apr. 17,
  pp.432 - 441, 2008
\item[]
  허재성, \textbf{박은천}, 이상현, 이성경, 민경원, 김홍진, 조지성,
  조봉호, 주석준, ``실물크기 구조물에 설치된 동조액체질량감쇠기의
  성능실험'', \emph{한국소음진동공학회 2008년도 춘계학술대회논문집} 2008
  Apr. 17, pp.449 - 457, 2008
\item[]
  윤경조, 이상현, \textbf{박은천}, 유은종, 민경원, ``실물크기 구조물의
  강제진동 실험을 통한 시스템 식별'', \emph{한국소음진동공학회 2007년도
  추계학술대회논문집} 2007 Nov. 15, pp.195 - 200, 2007
\item[]
  허재성, 이성경, 이상현, \textbf{박은천}, 김홍진, 조봉호, 조지성,
  김동영, 민경원, ``실시간 하이브리드 진동대 실험법에 의한 양방향 TLMD의
  풍응답 제어성능평가'', \emph{한국소음진동공학회 2007년도
  추계학술대회논문집} 2007 Nov. 15, pp.189 - 194, 2007
\item[]
  허재성, 김홍진, 조봉호, 조지성, \textbf{박은천}, 이상현, 이성경,
  김동영, 민경원, ``TLCD와 고무패드형 TMD를 이용한 2방향 TLMD의
  성능평가실험'', \emph{한국소음진동공학회 2007년도 추계학술대회논문집}
  2007 Nov. 15, pp.465 - 470, 2007
\item[]
  \textbf{박은천}, 이상현, 민경원, 강경수, ``ATMD를 이용한 건축 구조물의
  풍응답 구현을 위한 가진시스템'', \emph{한국소음진동공학회 2007년도
  추계학술대회논문집} 2007 Nov. 15, pp.210 - 215, 2007
\item[]
  \textbf{박은천}, 이상현, 민경원, 강경수, ``풍하중 구현 및 내풍특성
  평가를 위한 선형질량 가진시스템 설계'', \emph{한국소음진동공학회
  2007년도 추계학술대회논문집} 2007 Nov. 15, pp.661 - 668, 2007
\item[]
  정희산, 이성경, textbf{박은천}, 민경원, ``실시간 하이브리드 실험법을
  이용한 동조액체기둥감쇠기가 설치된 구조물의 지진응답 제어성능 평가'',
  \emph{한국소음진동공학회 2007년도 추계학술대회논문집} 2007 Nov. 15,
  pp.669 - 673, 2007
\item[]
  \textbf{박은천}, 이성경, 이헌재, 최강민, 문석준, 정형조, 정희산,
  민경원, ``실시간 하이브리드 실험법을 이용한 대형 MR감쇠기의 준능동
  제어알고리즘 성능 비교'', \emph{한국소음진동공학회 2007년도
  추계학술대회논문집} 2007 Nov. 15, pp.648 - 654, 2007
\item[]
  정희산, 이성경, \textbf{박은천}, 민경원, 이헌재, 최강민, 문석준,
  정형조, ``실시간 하이브리드 실험법을 이용한 대형 MR감쇠기의 제진
  성능평가'', \emph{한국소음진동공학회 2007년도 추계학술대회논문집} 2007
  Nov. 15, pp.655 - 660, 2007
\item[]
  이성경, 이상현, 민경원, \textbf{박은천}, 우성식, 정란, 윤경조,
  ``하이브리드 실험법을 이용한 TLD가 설치된 건물의 지진응답 제어'',
  \emph{한국지진공학회 2006년도 학술발표회 논문집 제10권} 2006 Mar. 17,
  pp.527 - 534, 2006
\item[]
  박지훈, 민경원, 문병욱, \textbf{박은천}, ``MR감쇠기가 설치된 구조물의
  등가선형 시스템에 대한 가진 특성의 영향'', \emph{한국지진공학회
  2006년도 학술발표회 논문집 제10권} 2006 Mar. 17, pp.503 - 510, 2006
\item[]
  이성경, 이상현, 민경원, \textbf{박은천}, 우성식, 정란, ``실시간
  하이브리드 실험법을 이용한 동조액체댐퍼가 설치된 건물의 진동제어'',
  \emph{한국전산구조공학회 2006년도 정기 학술대회 논문집} 2006 Apr. 01,
  pp.256 - 263, 2006
\item[]
  이상현, 민경원, 이명규, \textbf{박은천}, ``복합모드형 소형 MR감쇠장치
  성능에 관한 실험적 연구'', \emph{한국지진공학회 2005년도 학술발표회
  논문집} 2005 Mar. 18, pp.461 - 468, 2005
\end{itemize}

\subsubsection{Research projects}
\begin{itemize}
\item[] 콘크리트궤도 도상/노반결함 상태평가 시스템 개발, 주관:한국철도공사 2016-2017
\item[] 도시 인프라 자산관리 플랫폼과 서비스 모델 개발, 주관:(주)승화기술정책연구소 2015-2016
\item[] M2M 기반 지하공간(지하철) 재난대응 대화형 스마트 네트워크 시스템 개발, 주관:(주)승화기술정책연구소 2015-2017
\item[]
  진동특성을 이용한 LCD 생산로봇 및 부품 사전 진단 시스템 개발, 주관:한국유지관리(주) 2011-2012
\item[]
  감지형앵커를 이용한 사면 보강 및 모니터링 기술 개발,
  주관:한국유지관리(주) 2010-2011
\item[]
  교량 및 지반보강을 위한 스마트 텐던 기술의 사업화를 위한 추가기술개발
  및 검증, 주관:한국유지관리(주) 2010
\item[]
  철골조 시설물의 붕괴를 방지하는 설치 용이한 경제적인 보강기구 개발,
  주관:단국대학교 산학협력단 2006
\item[]
  다자유도 진동대 및 가력기를 이용한 2방향 동조식 액체형 댐퍼의 풍응답
  저감 연구, 주관:단국대학교 20080901 \textasciitilde{} 20110831
\item[]
  대형 비탄성구조물 분산 제어시스템 설계기술 개발, 주관:단국대학교
  20070301 \textasciitilde{} 20100228
\item[] 모듈러 유닛 구조물 구조성능 평가 및 운송기술 개발, 주관:(사)대한건축학회 2006-2007
\item[] 초대형 구조물의 내풍 및 내진성능향상을 위한 준능동 제어시스템 개발(The Development of a Semi-active Control System for Large-scale Structures under Wind and Seismic Loads), 주관:단국대학교 2005-2006
\end{itemize}

\subsubsection*{Lecture}

\begin{tabularx}{\textwidth}{p{4\parindent}XX}
    2011.09.29 & 망조정과 외란보정기법의 RTK GNSS를 적용한 고층구조물의 거푸집 연직도 관리기술 & 서울시 \\
    2012.09.01--2013.02.28 & 공학수치해석 (Numerical Analysis) & 단국대학교 (Dankook University)\\
\end{tabularx}

\subsection*{Intellectual property}

\subsubsection*{ - NT}

\paragraph{망조정과 외란보정기법의 RTK GNSS을 적용한 고층 구조물의 거푸집 연직도 관리기술}

\begin{itemize}
\item[]
  고시 : \href{http://www.kaia.re.kr/portal/newtec/view.do?searchCnd=1\&searchWrd=\&menuNo=200075\&frApntYear=\&toApntYear=\&pageUnit=10\&frApntNo=\&toApntNo=\&cate1=\&cate2=\&cate3=\&tecCat1=\&tecCat2=\&tecCat3=\&newtecCat1=\&newtecCat2=\&newtecCat3=\&dvlprNm=\%ED\%95\%9C\%EA\%B5\%AD\%EC\%9C\%A0\%EC\%A7\%80\%EA\%B4\%80\%EB\%A6\%AC\&ordDvs=\&pageIndex=1\&apntNo=625\&frMenu=list}{국토해양부고시
  제2011-313호}
\item[]
  지정번호 : 제625호
\item[]
  명칭 : 망조정과 외란보정기법의 RTK GNSS을 적용한 고층 구조물의 거푸집 연직도 관리기술
\item[]
  기술분야 : 건축시공
\item[]
  개발법인 : 한국유지관리(주), 롯데건설(주)
\item[]
  개발참여자 : 박순전, 석원균, 정원백, 최준성, 김민수, 이중엽, \emph{\textbf{박은천}}
\end{itemize}

\subsubsection{ - Patents}

\paragraph{철도차량의 승차감 분석 시스템(System for ride comfort analysis of Railway vehicle)}

\begin{itemize}
\item[]
  출원번호 : 1020100137577
\item[]
  등록번호 : 1012501040000
\item[]
  출원인 : 한국유지관리 주식회사, 코레일공항철도 주식회사
\item[]
  출원일자 : 2010.12.29
\item[]
  등록일 자: 2013.03.27
\item[]
  공개일자 : 2012.07.09
\item[]
  발명자 : 최준성, 박수열, \emph{\textbf{박은천}}, 김성용, 조한권
\end{itemize}

\paragraph{철도차량의 주행 안정성 분석 시스템(System for driving
stability analysis of Railway
vehicle)}

\begin{itemize}
\item[]
  출원번호 : 1020110068195
\item[]
  등록번호 : 1012590880000
\item[]
  출원인 : 한국유지관리 주식회사
\item[]
  출원일자 : 2011.07.11
\item[]
  등록일자 : 2013.04.23
\item[]
  공개일자 : 2013.01.21
\item[]
  발명자 : 최준성, 강형구, \emph{\textbf{박은천}}, 김만철, 유원희
\end{itemize}

\paragraph{이상진단 사전감시 방법(Method for preliminary surveillance of
failure
diagnosis)}

\begin{itemize}
\item[]
  출원번호 : 1020120138404
\item[]
  출원인 : 한국유지관리 주식회사
\item[]
  출원일자 : 2012.11.30
\item[]
  공개일자 : 2014.06.13
\item[]
  발명자 : 최준성, 임공철, 이은찬, \emph{\textbf{박은천}}
\end{itemize}

\paragraph{유에스엔 기반 지능형 교량 모니터링 및 안전성 평가
시스템(System for intelligent monitoring and safety evaluation of bridge
based on
USN)}

\begin{itemize}
\item[]
  출원번호 : 10-2011-0026810
\item[]
  출원인 : 한국유지관리 주식회사
\item[]
  출원일자 : 2011.03.25
\item[]
  공개일자 : 2012.10.17
\item[]
  발명자 : 최준성, \emph{\textbf{박은천}}, 윤종구
\end{itemize}

\section*{Relevant Skills}
% \subsection*{High Level}
% \begin{itemize}
%   \item[] Dynamics of structure, Signal processing, Structural analysis, Data analytics, Algorithms, System Identification, Feed-back control system
% \end{itemize}
% \subsection*{Mid Level}
% \begin{itemize}
%   \item[] Structural design, Adaptive control \& filter design, Optimal control design, Neural network, Convolutional neural network, Recurrent neural network, Computer vision, Image processing
% \end{itemize}
\subsection*{Computer Programming}
\begin{itemize}
  \item[] Mathworks MATLAB, Mathworks Simulink, National Instruments LabVIEW, Visual Studio (C\#, C++)
  \item[] Python, Node.js, Javascript, SQL, NoSQL(MongoDB), Shell script of Linux(Debian)
\end{itemize}

\section*{Referees}
\begin{tabularx}{\textwidth}{p{4\parindent}X}
    \textbf{Kyung-Won Min, Ph.D.} & : Professor, Advisor \\
    & Dept. of Architectural Engineering of Dankook University \\
    & E-mail : kwmin@dankook.ac.kr \\
    & Phone : +82-31-8005-3734 \\
    \\
    \textbf{Sang-Hyun Lee, Ph.D.} & : Professor \\
    & Dept. of Architectural Engineering of Dankook University \\
    & E-mail : lshyun00@dankook.ac.kr \\
    & Phone : +82-31-8005-3735 \\
    \\
\end{tabularx}
% \begin{itemize}
%   \item[] \textbf{Kyung-Won Min, Ph.D.} : Professor, Advisor \hfill Phone : +82-31-8005-3734 \\
%   Dept. of Architectural Engineering of Dankook University \hfill E-mail : kwmin@dankook.ac.kr
%   \item[] \textbf{Sang-Hyun Lee, Ph.D.} : Professor \hfill Phone : +82-31-8005-3735 \\
%   Dept. of Architectural Engineering of Dankook University \hfill E-mail : lshyun00@dankook.ac.kr
% \end{itemize}
