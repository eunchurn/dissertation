%!TEX program = xelatex

% Name           : hsrm-beamer-minimal.sty
% Author         : Benjamin Weiss (benjamin.weiss@kreatiefton.de)
% Version        : 0.2
% Created on     : 08.05.2013
% Last Edited on : 24.03.2014
% Copyright      : Copyright (c) 2013 by Benjamin Weiss. All rights reserved.
% License        : This file may be distributed and/or modified under the
%                  GNU Public License.
% Description    : HSRM beamer theme minimal example.

\documentclass[usepdftitle=false]{beamer}

\usepackage[main=english,german]{babel}
\newcommand\Fontvi{\fontsize{6}{7.2}\selectfont}
\usetheme{hsrm}
\usepackage{graphicx}
\usepackage{subfigure}
\usepackage{amsbsy}
\usepackage{amssymb}
\usepackage{amsmath}
\usepackage{tabularx}
\usepackage{booktabs}
%\usepackage{floatrow}
%\usepackage{multirow}
%\usepackage{multicol}
\usepackage{enumerate}
\setbeamerfont{caption}{size=\scriptsize
}%\floatsetup[table]{capposition=top}
\newcommand{\matr}[1]{\mathbf{#1}}
\newenvironment{bsmallmatrix}
  {\left[\begin{smallmatrix}}
  {\end{smallmatrix}\right]}
\newenvironment{psmallmatrix}
  {\left\{\begin{smallmatrix}}
  {\end{smallmatrix}\right\}}
\newenvironment{psmatrix}
  {\left\{\begin{matrix}}
  {\end{matrix}\right\}}
\newcommand{\code}[1]{\texttt{#1}}
\newcolumntype{b}{X}
\newcolumntype{s}{>{\hsize=.5\hsize}X}
%\includeonlyframes{current}

\title{비선형 감쇠장치가 설치된 건축구조물의 내진 및 내풍 성능 평가를 위한 하이브리드 실험기법 및 가진시스템 설계}
\subtitle{Hybrid Experimental Method and Excitation System Design for Seismic and Wind-resistance Performance Evaluation of Building Structures with Nonlinear Dampers}
\author{박은천}
\institute{단국대학교 건축대학 {\Medium 지도교수 : 민경원}}
\date{\today}

\begin{document}

\maketitle

\section*{Contents}
\begin{frame}
	\frametitle{Contents}
	\tableofcontents[hideallsubsections]
\end{frame}

\section{Introduction}

\begin{frame}{Introduction}
\begin{figure}[ht]
\centering
\includegraphics[width=1\textwidth] {figure/beamer-1.eps}
\end{figure}
\end{frame}

\begin{frame}{Introduction}
\begin{figure}[ht]
\centering
\includegraphics[width=1\textwidth] {figure/beamer-2.eps}
\end{figure}
\end{frame}

\begin{frame}{Introduction}
\begin{figure}[ht]
\centering
\includegraphics[width=1\textwidth] {figure/beamer-3.eps}
\end{figure}
\end{frame}

\begin{frame}{Introduction}
\begin{figure}[ht]
\centering
\includegraphics[width=1\textwidth] {figure/beamer-4.eps}
\end{figure}
\end{frame}

\begin{frame}{Introduction}
\begin{figure}[ht]
\centering
\includegraphics[width=1\textwidth] {figure/subject.eps}
\label{fig:subject}
\end{figure}
\end{frame}




\section{Publications}

\begin{frame}{International Journal (국제저널 SCI급)}
\Fontvi
\begin{itemize}
\item \textbf{Eunchurn Park}, Kyung-Won Min, Sung-Kyung Lee, Sang-Hyun Lee, Heon-Jae Lee, Seok-Jun Moon, Hyung-Jo Jung, “\textbf{Real-time Hybrid Test on a Semi-actively Controlled Building Structure Equipped with Full-scale MR Dampers}”, Journal of Intelligent Material Systems and Structures; 21(18):1831-1850. DOI:10.1177/1045389X10390253 · 2.17 Impact Factor
\item Heon-Jae Lee, Hyung-Jo Jung, Seok-Jun Moon, Sung-Kyung Lee, \textbf{Eun-Churn Park}, Kyung-Won Min, “\textbf{Experimental Investigation of MR Damper-based Semiactive Control Algorithms for Full-scale Five-story Steel Frame Building}”, Journal of Intelligent Material Systems and Structures; 21(10):1025-1037. DOI:10.1177/1045389X10374162 · 2.17 Impact Factor
\item Jae-Sung Heo, Sung-Kyung Lee, \textbf{Eunchurn Park}, Sang-Hyun Lee, Kyung-Won Min, Hongjin Kim, Jiseong Jo, Bong-Ho Cho, “\textbf{Performance test of a tuned liquid mass damper for reducing bidirectional responses of building structures}”, The Structural Design of Tall and Special Buildings; 18(7):789 - 805. DOI:10.1002/tal.486 · 0.83 Impact Factor
\item \textbf{Eun-Churn Park}, Sang-Hyun Lee, Kyung-Won Min, Lan Chung, Sung-Kyung Lee, Seung-Ho Cho, Eunjong Yu, Kyung-Soo Kang, “\textbf{Design of an actuator for simulating wind-induced response of a building structure}”, SMART STRUCTURES AND SYSTEMS; 4(1). DOI:10.12989/sss.2008.4.1.085 · 1.16 Impact Factor
\item Sung-Kyung Lee, \textbf{Eun Churn Park}, Kyung-Won Min, Ji-Hun Park, “\textbf{Real-time substructuring technique for the shaking table test of upper substructures}”, Engineering Structures; 29(9):2219-2232. DOI:10.1016/j.engstruct.2006.11.013 · 1.77 Impact Factor
\item Sung-Kyung Lee, \textbf{Eun Churn Park}, Kyung-Won Min, Sang-Hyun Lee, Ji-Hun Park, “\textbf{Experimental implementation of a building structure with a tuned liquid column damper based on the real-time hybrid testing method}”, Journal of Mechanical Science and Technology; 21(6):885-890. DOI:10.1007/BF03027063 · 0.70 Impact Factor
\item Sung-Kyung Lee, \textbf{Eun Churn Park}, Kyung-Won Min, Sang-Hyun Lee, Lan Chung, Ji-Hun Park, “ \textbf{Real-time hybrid shaking table testing method for the performance evaluation of a tuned liquid damper controlling seismic response of building structures}”, Journal of Sound and Vibration; DOI:10.1016/j.jsv.2006.12.006 · 1.86 Impact Factor
\end{itemize}
\end{frame}

\begin{frame}{Domestic Journal (국내저널 KCI등재)}
\Fontvi
\begin{itemize}
\item 민경원, 이성경, 박은천, “진동대실험에 의한 동조액체기둥감쇠기의 동적특성”, 한국소음진동공학회논문집 (Transactions of the Korean society for noise and vibration engineering) v.19 no.6 = no.147, pp.620 - 627, 2009, 1598-2785
\item 박은천, 이성경, 이헌재, 문석준, 정형조, 민경원, “대형 MR감쇠기가 설치된 건축구조물의 실시간 하이브리드 실험 및 준능동 알고리즘 적용”, 한국전산구조공학회논문집 (Journal of the computational structural engineering institute of Korea) v.21 no.5, pp.465 - 474, 2008, 1229-3059
\item 이성경, 민경원, 박은천, “TMD와 TLCD를 이용한 2방향 감쇠기의 동적특성”, 한국전산구조공학회논문집 (Journal of the computational structural engineering institute of Korea) v.21 no.6, pp.589 - 596, 2008, 1229-3059
\item 허재성, 이성경, 박은천, 이상현, 김홍진, 조지성, 조봉호, “실시간 하이브리드 진동대 실험법에 의한 양방향 TLMD의 진동제어 성능평가”, 한국소음진동공학회논문집 (Transactions of the Korean society for noise and vibration engineering) v.18 no.5 = no.134, pp.485 - 495, 2008, 1598-2785
\item 허재성, 박은천, 이상현, 이성경, 김홍진, 조봉호, 조지성, 김동영, 민경원, “건축구조물의 2방향 진동제어를 위한 동조액체질량감쇠기”, 한국소음진동공학회논문집 (Transactions of the Korean society for noise and vibration engineering) v.18 no.3 = no.132, pp.345 - 355, 2008, 1598-2785
\item 허재성, 박은천, 이성경, 이상현, 김홍진, 조지성, 조봉호, 주석준, 민경원, “실물크기 구조물에 설치된 동조액체질량감쇠기의 성능실험”, 한국전산구조공학회논문집 (Journal of the computational structural engineering institute of Korea) v.21 no.2, pp.161 - 168, 2008, 1229-3059
\item 윤경조, 박은천, 이헌재, 문석준, 민경원, 정형조, 이상현, “준능동 MR감쇠기가 설치된 실물크기 구조물의 분산제어 알고리즘 성능평가”, 한국소음진동공학회논문집 (Transactions of the Korean society for noise and vibration engineering) v.18 no.2 = no.131, pp.255 - 262, 2008, 1598-2785
\end{itemize}
\end{frame}

\begin{frame}{Domestic Journal (국내저널 KCI등재)}
\Fontvi
\begin{itemize}
\item 박은천, 이성경, 윤경조, 정희산, 이헌재, 최강민, 문석준, 정형조, 민경원, “실시간 하이브리드 실험법을 이용한 대형 MR감쇠기의 제진 성능평가”, 한국소음진동공학회논문집 (Transactions of the Korean society for noise and vibration engineering) v.18 no.1 = no.130, pp.131 - 138, 2008, 1598-2785
\item 박은천, 민경원, 정란, 강경수, 이상현, “건축구조물의 풍하중 구현 및 풍특성 평가를 위한 가진시스템 설계”, 한국전산구조공학회논문집 (Journal of the computational structural engineering institute of Korea) v.20 no.6, pp.769 - 778, 2007, 1229-3059
\item 이성경, 민경원, 박은천, “진동대를 이용한 구조물의 하이브리드 실험”, 대한건축학회논문집 (Journal of the architectural institute of Korea : Structure \& construction) / 구조계 v.22 no.5 = no.211, pp.57 - 63, 2006, 1226-9107
\item 이상현, 박은천, 윤경조, 이성경, 유은종, 민경원, 정란, 민정기, 김영찬, “실물 크기 구조물의 강제진동실험 및 지진응답 모사를 위한 HMD제어기 설계”, 한국지진공학회논문집 (Journal of the earthquake engineering society of Korea) v.10 no.6 = no.52, pp.103 - 114, 2006, 1226-525x
\item 이성경, 박은천, 이상현, 정란, 우성식, 민경원, “실시간 하이브리드 진동대 실험법을 이용한 TLD 제어성능의 실험적 검증”, 한국전산구조공학회논문집 (Journal of the computational structural engineering institute of Korea) v.19 no.4 = no.74, pp.419 - 427, 2006, 1229-3059
\end{itemize}
\end{frame}



\section{RT-HYTEM}
\begin{frame}{Real-Time Hybrid Testing Method}
\begin{figure}[ht]
\centering
\subfigure{
\includegraphics[width=0.5\textwidth] {figure/2_1a.eps}
}\hfill
\subfigure{
\includegraphics[width=0.5\textwidth] {figure/2_1b.eps}
}
\end{figure}
\end{frame}


\begin{frame}{Dynamic equilibrium in experimental structure}
\begin{figure}[ht]
\setcounter{subfigure}{0}
\centering\subfigure[Whole structure]{
   \includegraphics[width=0.2\textwidth] {figure/2-3a.eps}
   \label{fig:2-3a}
 }\hfill
 \subfigure[Separation of whole structure]{
   \includegraphics[width=0.3\textwidth] {figure/2-3b.eps}
   \label{fig:2-3b}
 }\hfill
 \subfigure[Experimental and numerical substructure]{
   \includegraphics[width=0.4\textwidth] {figure/2-3c.eps}
   \label{fig:2-3c}
 }
\end{figure}
\end{frame}


\begin{frame}{Overall view of experimental system}
\begin{figure}[ht]
\centering
\includegraphics[width=1\textwidth] {figure/2-5.eps}
\end{figure}
\end{frame}


\begin{frame}{Schematic diagram of experimental system}
\begin{figure}[ht]
\centering
\includegraphics[width=1\textwidth] {figure/2-6.eps}
\end{figure}
\end{frame}

\begin{frame}{Flow chart of the experimental system controller}
\begin{figure}[ht]
\centering
\includegraphics[width=1\textwidth] {figure/2-8.eps}
\end{figure}
\end{frame}


\begin{frame}{Interfacing Force}
\begin{align}
i(t)=&-m_{E(1)}\ddot{Y}_{E(1)}(t)-m_{E(2)}\ddot{Y}_{E(2)}(t) \label{eq:2-19} \\
\ddot{Y}_{N(3)}(t)=&-c_{N(3)}\left(\dot{Y}_{N(3)}-\dot{Y}_{N(2)}\right) -k_{N(3)}\left(Y_{N(3)}-Y_{N(2)}\right)+i(t) \label{eq:2-20}
\end{align}

\begin{figure}[ht]
\centering
\includegraphics[width=0.8\textwidth] {figure/2-11.eps}
\end{figure}
\end{frame}


\begin{frame}{Comparisons of results}
\begin{figure}[ht]
\centering
\setcounter{subfigure}{0}
\subfigure[Time domain]{
   \includegraphics[width=0.4\textwidth] {figure/2-12a.eps}
   \label{fig:2-12a}
 }
 \subfigure[Frequency domain]{
   \includegraphics[width=0.4\textwidth] {figure/2-12b.eps}
   \label{fig:2-12b}
 }
%\caption{Comparisons of results measured from the experiment without feedback and those calculated from numerical analysis.}
\label{fig:2-12}
\end{figure}
\end{frame}

\begin{frame}{Spectrogram comparison of results}
\begin{figure}[ht]
\centering
\setcounter{subfigure}{0}
 \subfigure[experiment without feedback]{
   \includegraphics[width=0.35\textwidth] {figure/2-13a.eps}
   \label{fig:2-13a}
 }\hfill
 \subfigure[numerical analysis measured]{
   \includegraphics[width=0.35\textwidth] {figure/2-13b.eps}
   \label{fig:2-13b}
 }\\
 \subfigure[experiment without feedback]{
   \includegraphics[width=0.35\textwidth] {figure/2-13c.eps}
   \label{fig:2-13c}
 }\hfill
 \subfigure[numerical analysis measured]{
   \includegraphics[width=0.35\textwidth] {figure/2-13d.eps}
   \label{fig:2-13d}
 }
\end{figure}
\end{frame}

\begin{frame}{RT-HYTEM for a TLD with Building Structure}
\begin{figure}[ht]
\centering
\includegraphics[width=1\textwidth] {figure/3-1.eps}
%\caption{Conceptual view of the real-time hybrid shaking table test}
%\label{fig:3-1}
\end{figure}
\end{frame}

\begin{frame}{RT-HYTEM for the Performance Evaluation of a TLD}
\begin{figure}[ht]
\centering
\includegraphics[width=1\textwidth] {figure/3-3.eps}
%\caption{Block diagram of the integrated controller for the real-time hybrid experimental system.}
\label{fig:3-3}
\end{figure}
\end{frame}

\begin{frame}{RT-HYTEM for the Performance Evaluation of a TLD}
\begin{figure}[!ht]
\centering
\setcounter{subfigure}{0}
\subfigure[Conventional shaking table test]{
   \includegraphics[width=0.45\textwidth] {figure/3-4a.eps}
   \label{fig:3-4a}
 }
 \subfigure[Real-time hybrid shaking table test]{
   \includegraphics[width=0.45\textwidth] {figure/3-4b.eps}
   \label{fig:3-4b}
 }
\end{figure}
\end{frame}


\begin{frame}{RT-HYTEM for the Performance Evaluation of a TLD}
\begin{figure}[!ht]
\centering
\setcounter{subfigure}{0}
 \subfigure[El Centro]{
   \includegraphics[width=0.4\textwidth] {figure/3-6a.eps}
   \label{fig:3-6a}
 }
 \subfigure[Hachinohe]{
   \includegraphics[width=0.4\textwidth] {figure/3-6b.eps}
   \label{fig:3-6b}
 }
 \subfigure[Mexico city]{
   \includegraphics[width=0.4\textwidth] {figure/3-6c.eps}
   \label{fig:3-6c}
 }
 \subfigure[Northridge]{
   \includegraphics[width=0.4\textwidth] {figure/3-6d.eps}
   \label{fig:3-6d}
 }
%\caption{Structural acceleration in the time domain measured from the conventional shaking table test of TLD-structure interaction system (dotted line : without control, solid line : with control)}
\label{fig:3-6}
\end{figure}
dotted line : without control, solid line : with control

\end{frame}

\begin{frame}{RT-HYTEM for the Performance Evaluation of a TLD}
\begin{figure}[!ht]
\centering
\setcounter{subfigure}{0}
 \subfigure[El Centro]{
   \includegraphics[width=0.25\textwidth] {figure/3-7a.eps}
   \label{fig:3-7a}
 }
 \subfigure[Hachinohe]{
   \includegraphics[width=0.25\textwidth] {figure/3-7b.eps}
   \label{fig:3-7b}
 }\\
 \subfigure[Mexico city]{
   \includegraphics[width=0.25\textwidth] {figure/3-7c.eps}
   \label{fig:3-7c}
 }
 \subfigure[Northridge]{
   \includegraphics[width=0.25\textwidth] {figure/3-7d.eps}
   \label{fig:3-7d}
 }
%\caption{Structural acceleration in the time domain measured from the conventional shaking table test of TLD-structure interaction system (dotted line : without control, solid line : with control)}
\label{fig:3-6}
\end{figure}
dotted line : without control, solid line : with control
\end{frame}



\begin{frame}{RT-HYTEM for the Performance Evaluation of a TLD}
\begin{figure}[!ht]
\centering
 \subfigure[El Centro]{
   \includegraphics[width=0.25\textwidth] {figure/3-9a.eps}
   \label{fig:3-9a}
 }
 \subfigure[Hachinohe]{
   \includegraphics[width=0.25\textwidth] {figure/3-9b.eps}
   \label{fig:3-9b}
 }\\
 \subfigure[Mexico city]{
   \includegraphics[width=0.25\textwidth] {figure/3-9c.eps}
   \label{fig:3-9c}
 }
 \subfigure[Northridge]{
   \includegraphics[width=0.25\textwidth] {figure/3-9d.eps}
   \label{fig:3-9d}
 }
\label{fig:3-9}
\end{figure}
dotted line : conventional shaking table test, solid line : real-time hybrid test
\end{frame}











\begin{frame}{RT-HYTEM for the Performance Evaluation of a TLCD}
\begin{figure}[ht]
\centering
\includegraphics[width=1\textwidth] {figure/4-1.eps}
%\caption{Concept of the real-time hybrid testing method (TLCD)}
\label{fig:4-1}
\end{figure}
\end{frame}

\begin{frame}{RT-HYTEM for the Performance Evaluation of a TLCD}
\begin{figure}[ht]
\centering
\setcounter{subfigure}{0}
\includegraphics[width=1\textwidth] {figure/4-3.eps}
%\caption{Controller for implementing the real-time hybrid testing method}
\label{fig:4-3}
\end{figure}
\end{frame}

\begin{frame}{RT-HSTTM for the Performance Evaluation of a TLCD}
\begin{figure}[!ht]
\centering
\setcounter{subfigure}{0}
\subfigure[conventional testing method]{
   \includegraphics[width=0.4\textwidth] {figure/4-4a.eps}
   \label{fig:4-4a}
 }
 \subfigure[real-time hybrid testing method]{
   \includegraphics[width=0.4\textwidth] {figure/4-4b.eps}
   \label{fig:4-4b}
 }
%\caption{Experimental view of a building with a TLCD}
\label{fig:4-4}
\end{figure}
\end{frame}

\begin{frame}{RT-HSTTM for the Performance Evaluation of a TLCD}
\begin{figure}[!ht]
\centering
\setcounter{subfigure}{0}
 \subfigure[El Centro Earthquake(time domain)]{
   \includegraphics[width=0.3\textwidth] {figure/4-5a.eps}
   \label{fig:4-5a}
 }
 \subfigure[El Centro Earthquake(frequency domain)]{
   \includegraphics[width=0.3\textwidth] {figure/4-5b.eps}
   \label{fig:4-5b}
 }\\
 \subfigure[Kobe Earthquake(time domain)]{
   \includegraphics[width=0.3\textwidth] {figure/4-5c.eps}
   \label{fig:4-5c}
 }
 \subfigure[Kobe Earthquake(frequency domain)]{
   \includegraphics[width=0.3\textwidth] {figure/4-5d.eps}
   \label{fig:4-5d}
 }
\end{figure}
\end{frame}

\begin{frame}{RT-HYTEM for the Performance Evaluation of a TLMD}
\begin{figure}[ht]
\centering
\includegraphics[width=0.6\textwidth] {figure/5-1.eps}
%\caption{Concept of a TLMD}
\label{fig:5-1}
\end{figure}
\end{frame}

\begin{frame}{RT-HYTEM for the Performance Evaluation of a TLMD}
\begin{figure}[ht]
\centering
\includegraphics[width=0.6\textwidth] {figure/5-2.eps}
%\caption{Photograph of the manufactured TLMD}
\label{fig:5-2}
\end{figure}
\Fontvi
\begin{table}[ht]
\centering
\begin{tabularx}{\textwidth}{@{}X|X|X@{}}
\toprule[1pt]\midrule[0.3pt]
Design parameter & TLCD control direction & TMD control direction\\ \hline
Siffness or liquid length & $0.98m$ (liquid length) & $1990N/m$ (stiffness)\\
Frequency (Hz) & $0.73$ & $0.82$\\
Mass (kg) & $35$ & $75$\\
\bottomrule
\end{tabularx}
\caption{Design parameters of a TLMD model}
\label{tab:5-3}
\end{table}
\end{frame}


\begin{frame}{RT-HYTEM for a TLMD with a Building Structure}
\begin{figure}[ht]
\centering
\setcounter{subfigure}{0}
\subfigure[Design of the controller]{
   \includegraphics[width=0.7\textwidth] {figure/5-16.eps}
}
\subfigure[MATLAB Simulink window Target]{
   \includegraphics[width=0.7\textwidth] {figure/5-18.eps}
}
\end{figure}
\end{frame}


\begin{frame}{RT-HYTEM for a TLMD with a Building Structure}
\begin{figure}[!ht]
\centering
\setcounter{subfigure}{0}
\subfigure[Disp. TMD dir.]{
   \includegraphics[width=0.3\textwidth] {figure/5-19.eps}
}
\subfigure[Acc. TMD dir.]{
   \includegraphics[width=0.3\textwidth] {figure/5-20.eps}
}
\subfigure[Disp. TMD dir.]{
   \includegraphics[width=0.3\textwidth] {figure/5-21.eps}
}
\subfigure[Disp. TLCD dir.]{
   \includegraphics[width=0.3\textwidth] {figure/5-22.eps}
}
\subfigure[Acc. TLCD dir.]{
   \includegraphics[width=0.3\textwidth] {figure/5-23.eps}
}
\subfigure[Disp. TLCD dir.t]{
   \includegraphics[width=0.3\textwidth] {figure/5-24.eps}
}
\end{figure}
\end{frame}

\section{Design of Controller}

\begin{frame}{Design of an Actuator for Simulating Wind Response}
\begin{figure}[!ht]
\centering
\includegraphics[width=0.8\textwidth] {figure/si.eps}
\end{figure}
\end{frame}

\begin{frame}{Design of an Actuator for Simulating Wind Response}
\begin{figure}[ht]
\centering
\includegraphics[width=0.8\textwidth] {figure/6-1.eps}
%\caption{Scheme of simulation of wind induced responses using LMS and ATMD}
\label{fig:6-1}
\end{figure}
Scheme of simulation of wind induced responses using LMS and ATMD
\end{frame}

\begin{frame}{Design of an Actuator for Simulating Wind Response}
Force of actuator
\begin{equation}
\begin{aligned}
\dot{\matr{z}} &=\matr{A}\matr{z}+\matr{B}_{f}f + \matr{B}_{u}u \\
y &= \matr{C}\matr{z}+\matr{D}_{f}f+\matr{D}_{u}u
\end{aligned}
\label{eq:6-1}
\end{equation}

\begin{equation}
\begin{aligned}
\matr{T}_{yf} &= \matr{Y}_{f}(s)\matr{F}(s)^{-1} = \matr{C}\left(s\matr{I}-\matr{A}\right)^{-1}\matr{B}_{f} \\
\matr{T}_{yu} &= \matr{Y}_{u}(s)\matr{U}(s)^{-1} = \matr{C}\left(s\matr{I}-\matr{A}\right)^{-1}\matr{B}_{u} \\
\end{aligned}
\label{eq:6-2}
\end{equation}

\begin{equation}
\begin{aligned}
\hat{\matr{U}}(s)&=\hat{\matr{T}}_{yu}^{-1}\hat{\matr{Y}}_{u}(s)\\
&=\hat{\matr{T}}_{yu}^{-1}\hat{\matr{Y}}_{f}(s)\\
&=\hat{\matr{T}}_{yu}^{-1}\hat{\matr{T}}_{yf}F(s)
\end{aligned}
\label{eq:6-4}
\end{equation}

\end{frame}


\begin{frame}{Design of an Actuator for Simulating Wind Response}
Filter and evelop function
\begin{equation}\label{eq:6-5}
\hat{\matr{U}}_{p}(\omega) = G(\omega)\hat{\matr{U}}(s)
\end{equation}

where,
\begin{align}
G(\omega) &= \frac{1-a_{co}}{2}cos \left( \frac{2\pi}{\omega_{2}-\omega_{1}}\omega \right) + \frac{1+a_{co}}{2}\label{eq:6-6}\\
a_{co} &=\left\{\begin{array}{lr} \omega < \omega_{1} &: 1 \\ \omega_{1} \leq \omega \leq \omega_{2} &: 0 \\ \omega > \omega_{2} &: 1\end{array} \right.
\label{eq:6-7}
\end{align}
\end{frame}

\begin{frame}{Design of an Actuator for Simulating Wind Response}
\begin{figure}[!ht]
\centering
\setcounter{subfigure}{0}
\subfigure[The shape of the band-stop filter]{
   \includegraphics[width=0.45\textwidth] {figure/6-2a.eps}
   \label{fig:6-2a}
 }
 \subfigure[The shape of the envelop function]{
   \includegraphics[width=0.45\textwidth] {figure/6-2b.eps}
   \label{fig:6-2b}
 }
%\caption{Exciter gain shape of the band-stop filter and the envelop function.}
\label{fig:6-2}
Exciter gain shape of the band-stop filter and the envelop function.
\end{figure}
\end{frame}

\begin{frame}{Design of an Actuator for Simulating Wind Response}
\begin{figure}[!ht]
\centering
\subfigure[Plan View of the 76-Story Building]{
   \includegraphics[width=0.3\textwidth] {figure/6-3a.eps}
   \label{fig:6-3a}\hfill
 }
 \subfigure[Elevation View of the Building.]{
   \includegraphics[width=0.3\textwidth] {figure/6-3b.eps}
   \label{fig:6-3b}
 }
\subfigure[Mode shapes of the Building.]{
   \includegraphics[width=0.3\textwidth] {figure/6-3c.eps}
   \label{fig:6-3c}\hfill
 }

%\caption{76th story benchmark model.}
\label{fig:6-3}
\end{figure}
76th story benchmark model.
\end{frame}

\begin{frame}{Design of an Actuator for Simulating Wind Response}
\begin{figure}[!ht]
\centering
\setcounter{subfigure}{0}
\subfigure[76th story acceleration response]{
   \includegraphics[width=0.8\textwidth] {figure/6-8a.eps}
   \label{fig:6-8a}\hfill
 }
 \subfigure[76th story displacement response]{
   \includegraphics[width=0.8\textwidth] {figure/6-8b.eps}
   \label{fig:6-8b}
 }

%\caption{Wind and LMS induced acceleration responses (when the target is 75th floor acceleration).}
\label{fig:6-8}
\end{figure}
Wind and LMS induced acceleration responses (when the target is 75th floor acceleration).
\end{frame}


\begin{frame}{Design of an Actuator for Simulating Wind Response}
\begin{figure}[!ht]
\centering
\setcounter{subfigure}{0}

 \subfigure[50th story acceleration response]{
   \includegraphics[width=0.8\textwidth] {figure/6-8c.eps}
   \label{fig:6-8c}\hfill
 }
 \subfigure[50th story displacement response]{
   \includegraphics[width=0.8\textwidth] {figure/6-8d.eps}
   \label{fig:6-8d}
 }

%\caption{Wind and LMS induced acceleration responses (when the target is 75th floor acceleration).}
\label{fig:6-8}
\end{figure}
Wind and LMS induced acceleration responses (when the target is 75th floor acceleration).
\end{frame}



\begin{frame}{Design of an Actuator for Simulating Wind Response}
\begin{figure}[!ht]
\centering
\setcounter{subfigure}{0}
 \subfigure[30th story acceleration response]{
   \includegraphics[width=0.8\textwidth] {figure/6-8e.eps}
   \label{fig:6-8e}\hfill
 }
 \subfigure[30th story displacement response]{
   \includegraphics[width=0.8\textwidth] {figure/6-8f.eps}
   \label{fig:6-8f}
 }
%\caption{Wind and LMS induced acceleration responses (when the target is 75th floor acceleration).}
\label{fig:6-8}
\end{figure}
Wind and LMS induced acceleration responses (when the target is 75th floor acceleration).
\end{frame}



\begin{frame}{Design of an Actuator for Simulating Wind Response}
ATMD Excitation
\begin{equation}\label{eq:6-10}
\matr{M}\matr{\ddot{x}}+\matr{C}\matr{\dot{x}}+\matr{K}\matr{x}+\matr{H}u = \eta\matr{W}
\end{equation}
Equation of motion of the building 
\begin{equation}\label{eq:6-11}
\begin{aligned}
\begin{bmatrix}\matr{M}_{s} & 0 \\ \matr{0} & m_{t}\end{bmatrix}
\begin{psmatrix}\matr{\ddot{x}}_{s} \\ \ddot{x}_{t}\end{psmatrix}+&
\begin{bmatrix}\matr{C}_{s}+c_{t}\matr{B}_{t}\matr{B}_{t}^{\top} & -c_{t}\matr{B}_{t} \\ -c_{t}\matr{B}_{t}^{\top} & c_{t}\end{bmatrix}
\begin{psmatrix}\matr{\dot{x}}_{s} \\ \dot{x}_{t}\end{psmatrix}+\\
&\begin{bmatrix}\matr{K}_{s}+k_{t}\matr{B}_{t}\matr{B}_{t}^{\top} & -k_{t}\matr{B}_{t} \\ -k_{t}\matr{B}_{t}^{\top} & k_{t}\end{bmatrix}
\begin{psmatrix}\matr{x}_{s} \\ x_{t}\end{psmatrix}  =
\begin{bmatrix}-\matr{B}_{t}\\1 \end{bmatrix}u
\end{aligned}
\end{equation}

\end{frame}


\begin{frame}{Design of an Actuator for Simulating Wind Response}
\begin{figure}[!ht]
\centering
\setcounter{subfigure}{0}
\subfigure[Transfer function of ATMD induced structure]{
   \includegraphics[width=0.45\textwidth] {figure/6-10a.eps}
   \label{fig:6-10a}\hfill
 }
 \subfigure[Frequency response of ATMD acceleration]{
   \includegraphics[width=0.45\textwidth] {figure/6-10b.eps}
   \label{fig:6-10b}
 }
 \subfigure[Frequency response of ATMD actuator force]{
   \includegraphics[width=0.45\textwidth] {figure/6-10c.eps}
   \label{fig:6-10c}\hfill
 }
 \subfigure[Time history of ATMD actuator force]{
   \includegraphics[width=0.45\textwidth] {figure/6-10d.eps}
   \label{fig:6-10d}
 }
%\caption{ATMD Excitation.}
\label{fig:6-10}
\end{figure}
\centering ATMD Excitation.
\end{frame}

\begin{frame}{Design of an Actuator for Simulating Wind Response}
\begin{figure}[!ht]
\centering
\setcounter{subfigure}{0}
\subfigure[76th story acceleration response]{
   \includegraphics[width=0.8\textwidth] {figure/6-11a.eps}
   \label{fig:6-11a}\hfill
 }
 \subfigure[76th story displacement response]{
   \includegraphics[width=0.8\textwidth] {figure/6-11b.eps}
   \label{fig:6-11b}
 }

%\caption{Wind and ATMD induced acceleration responses (when the target is 75th floor acceleration).}
\label{fig:6-11}
\end{figure}
Wind and ATMD induced acceleration responses (when the target is 75th floor acceleration).
\end{frame}

\begin{frame}{Design of an Actuator for Simulating Wind Response}
\begin{figure}[!ht]
\centering
\setcounter{subfigure}{0}

 \subfigure[50th story acceleration response]{
   \includegraphics[width=0.8\textwidth] {figure/6-11c.eps}
   \label{fig:6-11c}\hfill
 }
 \subfigure[50th story displacement response]{
   \includegraphics[width=0.8\textwidth] {figure/6-11d.eps}
   \label{fig:6-11d}
 }

%\caption{Wind and ATMD induced acceleration responses (when the target is 75th floor acceleration).}
\label{fig:6-11}
\end{figure}
Wind and ATMD induced acceleration responses (when the target is 75th floor acceleration).
\end{frame}



\begin{frame}{Design of an Actuator for Simulating Wind Response}
\begin{figure}[!ht]
\centering
\setcounter{subfigure}{0}

 \subfigure[30th story acceleration response]{
   \includegraphics[width=0.8\textwidth] {figure/6-11e.eps}
   \label{fig:6-11e}\hfill
 }
 \subfigure[30th story displacement response]{
   \includegraphics[width=0.8\textwidth] {figure/6-11f.eps}
   \label{fig:6-11f}
 }
%\caption{Wind and ATMD induced acceleration responses (when the target is 75th floor acceleration).}
\label{fig:6-11}
\end{figure}
Wind and ATMD induced acceleration responses (when the target is 75th floor acceleration).
\end{frame}



\begin{frame}{Forced Vibration Test of Real-scaled Building}
\begin{figure}[!ht]
\centering
\setcounter{subfigure}{0}
\subfigure[minor axis]{
   \includegraphics[width=0.18\textwidth] {figure/7-1a.eps}
   \label{fig:7-1a}\hfill
 }
 \subfigure[major axis]{
   \includegraphics[width=0.18\textwidth] {figure/7-1b.eps}
   \label{fig:7-1b}
 }
%\caption{Elevation view of the target structure.}
\label{fig:7-1}
\end{figure}
Elevation view of the target structure.
\end{frame}

\begin{frame}{Forced Vibration Test of Real-scaled Building}
\begin{figure}[ht]
\centering
\includegraphics[width=1\textwidth] {figure/7-3.eps}
%\caption{Schematic diagram of the field measurement, data acquisition and exciting system.}
\label{fig:7-3}
\end{figure}
Schematic diagram of the field measurement, data acquisition and exciting system.
\end{frame}

\begin{frame}{Forced Vibration Test of Real-scaled Building}
\begin{figure}[!ht]
\centering
\setcounter{subfigure}{0}
\subfigure[The measurement and data acquisition system]{
   \includegraphics[width=0.45\textwidth] {figure/7-4a.eps}
   \label{fig:7-4a}\hfill
 }
 \subfigure[The accelerometer installation]{
   \includegraphics[width=0.45\textwidth] {figure/7-4b.eps}
   \label{fig:7-4b}
 }
%\caption{Installation pictures of the measurement, data acquisition and exciting system.}
\label{fig:7-4}
\end{figure}
Installation pictures of the measurement, data acquisition and exciting system.
\end{frame}

\begin{frame}{Forced Vibration Test of Real-scaled Building}
\begin{figure}[ht]
\centering
\includegraphics[width=0.8\textwidth] {figure/7-5.eps}
%\caption{The transfer function from the absolute acceleration of the HMD to those of the structure.}
\label{fig:7-5}
\end{figure}
The transfer function from the absolute acceleration of the HMD to those of the structure.
\Fontvi
\begin{table}[ht]
\centering
\begin{tabularx}{\textwidth}{s|b|b}
\toprule[1pt]\midrule[0.3pt]
Modes & Frequency (Hz) \& COV (\%) & Damping ratio (\%) \& COV (\%)\\ \midrule[0.3pt]
1&0.52(1.82) & 1.46(14.42)\\
2&1.73(0.13) & 2.71(9.40)\\
3&2.94(0.10) & 3.54(1.93)\\
4&4.14(0.08) & 1.72(1.70)\\
5&5.36(0.23) & 3.84(3.56)\\
\bottomrule
\end{tabularx}
%\caption{Identified natural frequencies and damping ratios}
\label{tab:7-2}
\end{table}
Identified natural frequencies and damping ratios
\end{frame}

\begin{frame}{Forced Vibration Test of Real-scaled Building}
\begin{equation}\label{eq:7-4}
\matr{K}\Phi = \matr{M}_{a}\Phi\Lambda, \Phi^{\top}\matr{M}_{a}\Phi = \matr{I}
\end{equation}

\begin{equation}\label{eq:7-5}
\matr{K}^{\top} = \matr{K}
\end{equation}

\begin{equation}\label{eq:7-6}
J=\frac{1}{2}\left\| \matr{N}^{-1}\left(\matr{K}-\matr{K}_{a}\right)^{-1}\right\|
\end{equation}

\begin{equation}\label{eq:7-7}
\Phi = \Phi_{m} \left[ \Phi_{m}^{\top}\matr{M}_{a}\Phi_{m}\right]^{-1/2}
\end{equation}

\begin{equation}\label{eq:7-8}
\begin{aligned}
\matr{K}&=\matr{K}_{a} - \matr{K}_{a}\Phi\Phi^{\top}\matr{M}_{a}-\matr{M}_{a}\Phi\Phi^{\top}\matr{K}_{a}\\
&+\matr{M}_{a}\Phi\Phi^{\top}\matr{K}_{a}\Phi\Phi^{\top}\matr{M}_{a} + \matr{M}_{a}\Phi\Lambda\Phi^{\top}\matr{M}_{a}
\end{aligned}
\end{equation}
\end{frame}


\begin{frame}{Forced Vibration Test of Real-scaled Building}
\begin{figure}[!ht]
\centering
\setcounter{subfigure}{0}
\subfigure[1st mode shape]{
   \includegraphics[width=0.40\textwidth] {figure/7-6a.eps}
   \label{fig:7-6a}\hfill
 }
 \subfigure[2nd mode shape]{
   \includegraphics[width=0.40\textwidth] {figure/7-6b.eps}
   \label{fig:7-6b}
 }
%\caption{The mode shape comparison of initial, the measured and the updated FE models.}
\label{fig:7-6}
\end{figure}
The mode shape comparison of initial, the measured and the updated FE models.

\begin{table}[ht]
\centering
\begin{tabularx}{\textwidth}{@{}X|X|X|X|X|X@{}}
\toprule[1pt]\midrule[0.3pt]
& 1st mode & 2nd mode & 3rd mode & 4th mode & 5th mode\\ \midrule[0.3pt]
initial & 0.5022 & 1.5623 & 2.74 & 3.91 & 4.83\\
measured& 0.5249 & 1.7578 & 2.94 & 3.67 & 5.38\\
updated & 0.5249 & 1.7578 & 2.95 & 3.67 & 5.38\\
\bottomrule
\end{tabularx}
%\caption{Natural frequencies (Hz) for the modal testing tower}
\label{tab:7-3}
\end{table}
Natural frequencies (Hz) for the modal testing tower
\end{frame}


\begin{frame}{Forced Vibration Test of Real-scaled Building}
\begin{equation}\label{eq:7-11}
\begin{aligned}
\matr{\dot{z}}&= \matr{A}\matr{z} + \matr{B}_{b}\ddot{u}_{b}+\matr{B}_{h}\ddot{u}_{h}\\
\matr{y}&=\matr{C}\matr{z}+\matr{D}_{b}\ddot{u}_{b}+\matr{D}_{h}\ddot{u}_{h}
\end{aligned}
\end{equation}

\begin{equation}\label{eq:7-12}
\begin{aligned}
\matr{T}_{h}&=\matr{Y}_{h}(s)\matr{U}_{h}(s)^{-1} = \matr{C}\left(s\matr{I}-\matr{A}\right)^{-1}\matr{B}_{h} + \matr{D}_{h}\\
\matr{T}_{b}&=\matr{Y}_{b}(s)\matr{U}_{b}(s)^{-1} = \matr{C}\left(s\matr{I}-\matr{A}\right)^{-1}\matr{B}_{b}
\end{aligned}
\end{equation}

\begin{equation}\label{eq:7-13}
\matr{U}_{h}(s) = \matr{T}_{h}^{-1}\matr{Y}_{h}(s) = \matr{T}_{h}^{-1}\matr{Y}_{b}(s) = \matr{T}_{h}^{-1}\matr{T}_{b}\matr{U}_{b}(s)
\end{equation}

\begin{equation}\label{eq:7-14}
\hat{U}_{h}(s) = \hat{T}_{h}^{-1}\hat{Y}_{h}(s) = \hat{T}_{h}^{-1}\hat{Y}_{b}(s) = \hat{T}_{h}^{-1}\hat{T}_{b}\hat{U}_{b}(s)
\end{equation}

\end{frame}

\begin{frame}{Forced Vibration Test of Real-scaled Building}
\begin{figure}[!ht]
\centering
\setcounter{subfigure}{0}
\subfigure[Definition of the transfer function of the HMD]{
   \includegraphics[width=0.8\textwidth] {figure/7-9a.eps}
   \label{fig:7-9a}\hfill
 }
 \subfigure[Compensation the dynamics of HMD using the inverse transfer function]{
   \includegraphics[width=0.8\textwidth] {figure/7-9b.eps}
   \label{fig:7-9b}
 }
%\caption{Schematic diagram of the HMD controller.}
\label{fig:7-9}
\end{figure}
Schematic diagram of the HMD controller.
\end{frame}

\begin{frame}{Forced Vibration Test of Real-scaled Building}
\begin{figure}[!ht]
\centering
\setcounter{subfigure}{0}
\subfigure[1st floor acceleration]{
   \includegraphics[width=0.7\textwidth] {figure/7-12a.eps}
   \label{fig:7-12a}\hfill
}
\subfigure[2nd floor acceleration]{
   \includegraphics[width=0.7\textwidth] {figure/7-12b.eps}
   \label{fig:7-12b}
}
\subfigure[3rd floor acceleration]{
   \includegraphics[width=0.7\textwidth] {figure/7-12c.eps}
   \label{fig:7-12c}\hfill
}

%\caption{Time history comparison of El Centro earthquake response in the analysis and experimental models. (when the target response is the 5th floor acceleration).}

\end{figure}
Time history comparison of El Centro earthquake response in the analysis and experimental models. (when the target response is the 5th floor acceleration).
\end{frame}


\begin{frame}{Forced Vibration Test of Real-scaled Building}
\begin{figure}[!ht]
\centering
%\setcounter{subfigure}{0}
\subfigure[4th floor acceleration]{
   \includegraphics[width=0.7\textwidth] {figure/7-12d.eps}
   \label{fig:7-12d}
}
\subfigure[5th floor acceleration]{
   \includegraphics[width=0.7\textwidth] {figure/7-12e.eps}
   \label{fig:7-12e}\hfill
}
\subfigure[1st floor displacement]{
   \includegraphics[width=0.7\textwidth] {figure/7-12f.eps}
   \label{fig:7-12f}
}
%\caption{Time history comparison of El Centro earthquake response in the analysis and experimental models. (when the target response is the 5th floor acceleration).}

\end{figure}
Time history comparison of El Centro earthquake response in the analysis and experimental models. (when the target response is the 5th floor acceleration).
\end{frame}


\begin{frame}{Full scaled MR-damper}
\begin{figure}[!ht]
\centering
\subfigure[Components of full scaled MR Damper]{
   \includegraphics[width=0.4\textwidth] {figure/n3-6a.eps}
   \label{fig:n3-6a}\hfill
}
\subfigure[Manufactured full scaled MR Damper]{
   \includegraphics[width=0.4\textwidth] {figure/n3-6b.eps}
   \label{fig:n3-6b}
}

\label{fig:n3-6}
\end{figure}
\end{frame}

\begin{frame}{Full scaled MR-damper}
\begin{figure}[!ht]
\centering
\includegraphics[width=0.6\textwidth] {figure/n3-6c.eps}
   \label{fig:n3-6c}

\label{fig:n3-6}
\end{figure}
Manufactured MR damper with capcity of 1.0 ton
\end{frame}



\begin{frame}{Full state variable feedback for semi-active algorithms}

\begin{equation}\label{eq:n3-13}
P = \lim_{t\to\infty} E\left(\left[ \mathbf{x} - \hat{\mathbf{x}} \right] \left[ \mathbf{x} - \hat{\mathbf{x}} \right]^{\top}\right)
\end{equation}

\begin{equation}\label{eq:n3-14}
\begin{aligned}
\hat{\mathbf{x}} &= \left(\mathbf{A}-\mathbf{LC}\right)\hat{\mathbf{x}} + \begin{bmatrix}\mathbf{L} & \mathbf{B}-\mathbf{LD}\end{bmatrix} \begin{psmatrix}\mathbf{y}\\\mathbf{u}\end{psmatrix} \\
\begin{psmatrix}\hat{\mathbf{y}}\\\hat{\mathbf{x}}\end{psmatrix} & = \begin{bmatrix}\mathbf{C} \\ \mathbf{I}\end{bmatrix}\hat{\mathbf{x}} + \begin{bmatrix}\mathbf{0} & \mathbf{D}\\ \mathbf{0} & \mathbf{0}\end{bmatrix}\begin{psmatrix}y \\ \mathbf{u}\end{psmatrix}
\end{aligned}
\end{equation}

\end{frame}

\begin{frame}{MR Damper Test}
\begin{figure}[ht]
\centering
\subfigure[Installation of MR damper and load cell]{
\includegraphics[width=0.4\textwidth] {figure/installed1.eps}
}
\subfigure[LVDT type displacement sensor (Midori America LP-19FB)]{
\includegraphics[width=0.4\textwidth] {figure/installed2.eps}
}
\end{figure}
MR damper installation
\end{frame}




\begin{frame}{MR Damper Optimal Passive Controlled Result}
\begin{figure}[!ht]
\centering
\includegraphics[width=0.6\textwidth] {figure/n3-12.eps}

\label{fig:n3-12}
\end{figure}
Normalized maximum displacement of passive control cases (first floor).
\end{frame}

\begin{frame}{MR Damper Optimal Passive Controlled Result}
\begin{figure}[!ht]
\centering
\includegraphics[width=0.6\textwidth] {figure/n3-13.eps}
\label{fig:n3-13}
\end{figure}
Normalized maximum acceleration of passive control cases.
\end{frame}



\begin{frame}{MR Damper Semi-active controlled Result}
\begin{figure}[!ht]
\centering
\includegraphics[width=0.6\textwidth] {figure/n3-14.eps}
\label{fig:n3-14}
\end{figure}
Normalized maximum displacement comparison of MR damper-based control systems.
\end{frame}

\begin{frame}{MR Damper Semi-active controlled Result}
\begin{figure}[!ht]
\centering
\includegraphics[width=0.6\textwidth] {figure/n3-15.eps}
\label{fig:n3-15}
\end{figure}
Normalized maximum acceleration comparison of MR damper-based control systems.
\end{frame}


\begin{frame}{MR Damper Test Time-history Result}
\begin{figure}[!ht]
\centering
\includegraphics[width=0.5\textwidth] {figure/n3-16.eps}
\label{fig:n3-16}
\end{figure}
Time history responses of displacement at the first floor under the El Centro earthquake.
\end{frame}

\begin{frame}{MR Damper Test Time-history Result}
\begin{figure}[!ht]
\centering
\includegraphics[width=0.5\textwidth] {figure/n3-17.eps}
\end{figure}
Time history responses of acceleration at the second floor under the Kobe earthquake.
\label{fig:n3-17}
\end{frame}






\section{RT-HYTEM with MR Damper}
%Real-time Hybrid Test on a Semi-actively Controlled Building Structure Equipped with Full-scale MR Dampers


\begin{frame}{RT-HYTEM Building Structure with MR Damper}
\begin{figure}[!ht]
\centering
\setcounter{subfigure}{0}
\subfigure[structural control system]{
   \includegraphics[width=0.2\textwidth] {figure/8-1a.eps}
   \label{fig:8-1a}\hfill
 }
\subfigure[experimental and numerical substructures]{
   \includegraphics[width=0.4\textwidth] {figure/8-1b.eps}
   \label{fig:8-1b}
}
\subfigure[implementation of RT-HYTEM]{
   \includegraphics[width=0.5\textwidth] {figure/8-1c.eps}
   \label{fig:8-1c}
}
%\caption{Conceptual view of RT-HYTEM for a building with an MR damper.}
\label{fig:8-1}
\end{figure}
Conceptual view of RT-HYTEM for a building with an MR damper.
\end{frame}

\begin{frame}{RT-HYTEM Building Structure with MR Damper}
\begin{figure}[!ht]
\centering
\includegraphics[width=0.45\textwidth] {figure/8-2.eps}
%\caption{Configuration of experimental system.}
\label{fig:8-2}
\end{figure}
Configuration of experimental system.
\end{frame}


\begin{frame}{RT-HYTEM Building Structure with MR Damper}
\begin{figure}[ht]
\centering
\includegraphics[width=0.7\textwidth] {figure/8-3.eps}
%\caption{Schematic view of experimental set-up: (a) building model installed with an MR damper, (b) UTM installed MR damper, and (c) experimental instrumentation.}
\label{fig:8-3}
\end{figure}
Schematic view of experimental set-up: (a) building model installed with an MR damper, (b) UTM installed MR damper, and (c) experimental instrumentation
\end{frame}

\begin{frame}{RT-HYTEM Building Structure with MR Damper}
\begin{figure}[H]
\centering
\includegraphics[width=0.7\textwidth] {figure/8-7.eps}
%\caption{Integrated controller for implementing RTHTM.}
\label{fig:8-7}
\end{figure}
Integrated controller for implementing RTHTM.
\end{frame}

\begin{frame}{RT-HYTEM Building Structure with MR Damper}
\begin{figure}[H]
\centering
\setcounter{subfigure}{0}
\subfigure[time history of force]{
   \includegraphics[width=0.9\textwidth] {figure/8-8a.eps}
   \label{fig:8-8a}
}
\subfigure[force-displacement relation]{
   \includegraphics[width=0.45\textwidth] {figure/8-8b.eps}
   \label{fig:8-8b}
}
\subfigure[force-velocity relation]{
   \includegraphics[width=0.42\textwidth] {figure/8-8c.eps}
   \label{fig:8-8c}
}
%\caption{Comparison between calculated and experimentally measured responses for the Bouc-Wen model.}
\label{fig:8-8}
\end{figure}
Comparison between calculated and experimentally measured responses for the Bouc-Wen model.
\end{frame}


\begin{frame}{RT-HYTEM Building Structure with MR Damper}
\begin{figure}[H]
\centering
\setcounter{subfigure}{0}
\subfigure[absolute acceleration responses]{
   \includegraphics[width=0.8\textwidth] {figure/8-9a.eps}
   \label{fig:8-9a}
}
%\caption{Comparison between calculated and experimentally measured responses under El Centro earthquake excitation.}
\label{fig:8-9}
\end{figure}
Comparison between calculated and experimentally measured responses under El Centro earthquake excitation.
\end{frame}


\begin{frame}{RT-HYTEM Building Structure with MR Damper}
\begin{figure}[H]
\centering
\subfigure[displacement responses]{
   \includegraphics[width=0.8\textwidth] {figure/8-9b.eps}
   \label{fig:8-9b}
}
%\caption{Comparison between calculated and experimentally measured responses under El Centro earthquake excitation.}
\label{fig:8-9}
\end{figure}
Comparison between calculated and experimentally measured responses under El Centro earthquake excitation.
\end{frame}


\begin{frame}{RT-HYTEM Building Structure with MR Damper}
\begin{figure}[H]
\centering
\setcounter{subfigure}{0}
\subfigure[force-displacement relation]{
   \includegraphics[width=0.45\textwidth] {figure/8-13a.eps}
   \label{fig:8-13a}
}
\subfigure[force-velocity relation]{
   \includegraphics[width=0.45\textwidth] {figure/8-13b.eps}
   \label{fig:8-13b}
}\\

%\caption{Identified Bouc-Wen parameters and the numerical model of an MR damper.}
\label{fig:8-13}
\end{figure}
Identified Bouc-Wen parameters and the numerical model of an MR damper.
\end{frame}


\begin{frame}{RT-HYTEM Building Structure with MR Damper}
\begin{figure}[H]
\centering
\setcounter{subfigure}{0}
\subfigure[shape parameter]{
   \includegraphics[width=0.3\textwidth] {figure/8-13c.eps}
   \label{fig:8-13c}
}
\subfigure[damping coefficient of an MR damper]{
   \includegraphics[width=0.3\textwidth] {figure/8-13d.eps}
   \label{fig:8-13d}
}
\subfigure[shape parameter $\alpha$]{
   \includegraphics[width=0.3\textwidth] {figure/8-13e.eps}
   \label{fig:8-13e}
}
%\caption{Identified Bouc-Wen parameters and the numerical model of an MR damper.}
\label{fig:8-13}
\end{figure}
Identified Bouc-Wen parameters and the numerical model of an MR damper.
\end{frame}



\begin{frame}{RT-HYTEM Building Structure with MR Damper}
\begin{figure}[H]
\centering
\includegraphics[width=1\textwidth] {figure/8-14.eps}
%\caption{Clipped-optimal and RT-HYTEM integrated controller.}
\label{fig:8-14}
\end{figure}
Clipped-optimal and RT-HYTEM integrated controller.
\begin{equation}\label{eq:8-21}
v_{i} = V_{\text{max}}H\left(\left\{f_{c_{i}}-f_{i}\right\}f_{i}\right)
\end{equation}
\end{frame}


\begin{frame}{RT-HYTEM Building Structure with MR Damper}
\begin{figure}[H]
\centering
\includegraphics[width=1\textwidth] {figure/8-15.eps}
%\caption{MHF and RT-HYTEM integrated controller.}
\label{fig:8-15}
\end{figure}
MHF and RT-HYTEM integrated controller.
\begin{equation}\label{eq:8-22}
N_{i}(t)=g_{i}|P\left[\Delta_{i}(t)\right]|
\end{equation}
\begin{equation}\label{eq:8-23}
f_{n_{i}}=\mu g_{i}|P\left[\Delta_{i}(t)\right]=g_{n_{i}}|P\left[\Delta_{i}(t)\right]
\end{equation}
\begin{equation}\label{eq:8-24}
v_{i} = V_{\text{max}}H\left(f_{n_{i}}-|f_{i}|\right)
\end{equation}
\end{frame}

\begin{frame}{RT-HYTEM Building Structure with MR Damper}
\begin{figure}[H]
\centering
\includegraphics[width=0.8\textwidth] {figure/8-16.eps}
%\caption{Experimental results, in the time domain under El Centro earthquake excitation at different applied currents.}
\label{fig:8-16}
\end{figure}
Experimental results, in the time domain under El Centro earthquake excitation at different applied currents
\end{frame}



\begin{frame}{RT-HYTEM Building Structure with MR Damper}
\begin{figure}[H]
\centering
\includegraphics[width=0.8\textwidth] {figure/8-17.eps}
%\caption{Experimental results, in the time domain under Northridge earthquake excitation at different applied currents.}
\label{fig:8-17}
\end{figure}
Experimental results, in the time domain under Northridge earthquake excitation at different applied currents
\end{frame}

\begin{frame}{RT-HYTEM Building Structure with MR Damper}
\begin{figure}[H]
\centering
\setcounter{subfigure}{0}
\subfigure[El Centro earthquake]{
   \includegraphics[width=0.4\textwidth] {figure/8-20a.eps}
   \label{fig:8-20a}
}
\subfigure[Kobe earthquake]{
   \includegraphics[width=0.4\textwidth] {figure/8-20b.eps}
   \label{fig:8-20b}
}
\subfigure[Northridge earthquake]{
   \includegraphics[width=0.6\textwidth] {figure/8-20c.eps}
   \label{fig:8-20c}
}
%\caption{Passive and semi-active RT-HYTEM experimental results(time domain).}
\label{fig:8-20}
\end{figure}
Passive and semi-active RT-HYTEM experimental results(time domain).
\end{frame}

\begin{frame}{RT-HYTEM Building Structure with MR Damper}
\begin{figure}[H]
\setcounter{subfigure}{0}
\subfigure[El Centro earthquake]{
   \includegraphics[width=0.4\textwidth] {figure/8-21a.eps}
   \label{fig:8-21a}
}
\hfill\subfigure[Kobe earthquake]{
   \includegraphics[width=0.4\textwidth] {figure/8-21b.eps}
   \label{fig:8-21b}
}\\
\subfigure[Northridge earthquake]{
   \includegraphics[width=0.4\textwidth] {figure/8-21c.eps}
   \label{fig:8-21c}
}
\hfill\subfigure{
   \includegraphics[width=0.2\textwidth] {figure/8-21d.eps}
}
%\caption{Passive and semi-active RT-HYTEM experimental results(frequency domain).}
\label{fig:8-21}
\end{figure}
Passive and semi-active RT-HYTEM experimental results(frequency domain).
\end{frame}



\section{Conclusion}

\begin{frame}{Conclusion}
\begin{itemize}
   \item 실시간 하이브리드 실험법과 동적 응답을 구현하는 가진시스템 설계법을 제안
   \item 부분구조 실험법을 수행하였으며, TLD와 TLCD 그리고 새로운 형태의 2방향 제어장치 TLMD 하이브리드 실험과 일반적인 진동대 실험을 비교
   \item 풍하중 구현을 위해 76층 벤치마크 건물을 목표로 수치해석적으로 검증
   \item 실물규모 5층 건물에서 유사 지진응답을 구현하는 실험을 수행
   \item 실물규모 5층 건물에 1톤급 MR감쇠기를 설치하고 Passive 및 준능동 알고리즘을 적용하여 유사지진 응답 실험을 수행
   \item 1톤급 실물크기의 MR감쇠기를 UTM에 장착하여 실시간 하이브리드 실험법으로 제어성능을 검증
\end{itemize}
\end{frame}



\begin{frame}{수정 사항}
\begin{itemize}
   \item Abstract와 Introduction 수정 중, 3장 4장을 다시 구성 중.
   \item 민경원 교수님 : Real-time Hybrid Testing Method, Real-time Substructure Test, Real-time Hybrid Shaking Table Test Method -> 용어 통일 (수정 중)
   \item 정란 교수님 : Real-time Hybrid Testing Method라는 용어를 쓰는지, Hybrid 라는 용어를 쓰는것이 맞는지 고려바람 -> 많은 문헌에서 실험 모델과 수치 해석 모델을 동시에 하는 Hybrid 실험을 언급함, Abstract 4장에 대한 설명이 이해하기 어려움 수정 바람 (수정 완료)
   \item 이상현 교수님 : 논문 제목 수정 (수정 완료), 용어 약자를 풀어쓰는 구문이 반복됨 (수정 중)
   \item 김준희 교수님 : Introduction에 전체 논문 구성을 설명하는 것이 중요, 3장 MR damper 실험을 했기 때문에 4장에서 RT-HYTEM이 다시 나오면 3장과 4장을 합치는 방법을 제안함.
\end{itemize}
\end{frame}

\end{document}

