\begin{center}
\noindent {\large \textbf{비선형 감쇠장치가 설치된 건축구조물의}} \\
\vspace*{0.3cm}
\noindent {\large \textbf{내진 및 내풍 성능 평가를 위한}} \\
\vspace*{0.3cm}
\noindent {\large \textbf{하이브리드 실험기법 및 가진시스템 설계}} \\

\end{center}
\begin{flushright}
박 은 천\\
건축공학과\\
단국대학교 대학원\\
\vspace{0.3cm}
지도교수 : 민 경 원
\end{flushright}

\setlength{\baselineskip}{1.5\baselineskip}
지진이나 바람하중과 같은 외부하중으로 인해 가진되는 건축구조물의 동적 응답 특성을 정확하게 식별하는 것은 건축구조물의 안전성과 사용성 평가 뿐만 아니라 내진 또는 내풍 설계에 사용되는 해석 모델의 검증에 필수적이다. 건축구조물의 해석 모델을 구축하기 위하여 정확한 입출력 관계를 설명하는 시스템 행렬을 구성하는 시스템 식별 분야에서는 입력이 주요 모드를 가진시키기에 충분한 에너지를 가져야하고 구조물의 정보를 포함하는 양질의 응답인 출력을 측정하는 것이 매우 중요하다. 그러나 내진 성능 평가를 위하여 실물 크기의 건물을 진동대위에 설치하여 실험하기에 시스템을 구축하기에 매우 어려우며, 내풍 성능 평가를 위하여 실물 크기의 고층 건축물의 실험실에서 풍동 테스트하는 방법은 거의 불가능하다. 따라서 일반적으로 상사법칙을 적용한 축소 모형의 실험체를 제작하여 실험하거나 비선형 혹은 비탄성거동이 예측되는 부분구조를 나누어 제작하고 정적 실험으로 대체한다. 또한 사용중인 실물 크기의 건물에 대해서는 편심가진기나 선형질량가진기등의 가진기를 설치하고 주요 모드를 조화 하중으로 가진하여 건축물의 동적 성능 평가를 하거나 시스템 식별을 위한 수단으로 사용된다. 그러나 정확한 내진 및 내풍 성능을 실험적으로 평가하기 위해서 지진하중의 특성과 바람하중의 특성이 모두 충분히 실험에 반영되도록 건축구조물을 가진해야 할 필요가 있다. 

따라서 본 논문은 구조계를 분할하여 실험하는 방식의 하이브리드 실험법과 실물 크기의 건축구조물을 가진하여 동적특성을 식별하기 위한 가진 시스템 설계 기법을 제안한다. 첫째로 하이브리드 실험법은 전체 구조계에서 비선형성이 강한 파트를 실험모델 그리고 선형거동 할 것이라 예측되는 파트를 해석모델로 두고 실시간으로 그 응답을 받아 실험하는 기법이다. 이러한 실험 방법은 실험모델과 해석모델의 인터페이스를 선정해야하며 해석모델의 경계면은 실험모델의 가진력으로 작용하도록 설계하는 실험 시스템을 구축하여야 한다. 본 논문은 진동대를 통해 구조계를 실험구조계와 해석구조계로 분리하여 하이브리드 실험을 하는 부분구조실험법과 비선형 진동제어장치인 동조액체감쇠기, 동조액체기둥감쇠기 그리고 동조액체질량감쇠기를 실험모델로 실험하고 건축구조몰을 해석모델로 계산하는 하이브리드 실험을 설계하고 실험하여 그 성능을 검증하였다. 더나아가 비선형성이 강한 1톤급 층간형 감쇠기인 MR 감쇠기를 UTM에 장착하여 MR 감쇠기가 설치된 건물의 내진성능 평가를 수행하고 MR감쇠기에 준능동 알고리즘을 적용하여 내진 제어성능을 실험적으로 평가하였다. 이는 본 실험 방법을 통해 불특정 비선형성을 가진 제어장치의 최적의 알고리즘 혹은 장치 자체의 성능을 실험적으로 찾아서 적용할 수 있는 가능성과 정량적 근거를 제시할 수 있었다. 두번째로 동적 응답을 구현하는 가진시스템 설계법은 실물크기 사용중 건물에서 선형질량가진기나 능동형질량감쇠기 등을 특정 층에 설치하고 유사지진 또는 유사 바람하중 응답을 구현하는 시스템이다. 본 논문에서는 풍하중 응답 구현을 위해 건물 모델과 풍하중 데이터가 있는 76층 벤치마크 건물을 대상으로 수치해석적으로 검증하였으며, 지진하중 구현을 위해 실물규모의 5층 건물의 HMD를 가진하여 지진하중을 구현하였다. 또한 이를 제어하는 준능동 감쇠장치인 1톤급 MR 감쇠기를 설치하여 실물규모 건물에서 MR감쇠기의 지진에 대한 준능동 제어 성능을 평가하였다. 이러한 기법은 실물크기 건축구조물의 내진 및 내풍성능 평가와 비선형 제어장치가 설치된 건물의 내진 및 내풍성능 또한 평가하는데 사용될 수 있다.
\\
\\
{\large\textbf{주요어:}}
하이브리드 실험, 부분구조 실험, 비선형 진동제어 장치, 동조 액체 감쇠기, 동조 액체기둥 감쇠기, 동조 액체 질량 감쇠기, 제어-구조 상호력, 가진기 동특성 보상, 능동 질량 감쇠기, 복합형 질량 감쇠기, 가진 제어기 설계, 진동대 실험, 76층 벤치마크 건축구조물, 강제 진동 실험, 시스템 식별, 유한요소 모델 업데이트, 자기유변유체 감쇠기, 준능동 제어 알고리즘.
\\
\\
학 번 : 72070621\\
\noindent\rule[2pt]{\textwidth}{0.5pt}
