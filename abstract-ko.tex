\begin{center}
\noindent {\large \textbf{비선형 감쇠장치가 설치된 건축구조물의}} \\
\vspace*{0.3cm}
\noindent {\large \textbf{내진 및 내풍 성능 평가를 위한}} \\
\vspace*{0.3cm}
\noindent {\large \textbf{하이브리드 실험기법 및 가진시스템 설계}} \\

\end{center}
\begin{flushright}
박 은 천\\
건축공학과\\
단국대학교 대학원\\
\vspace{0.3cm}
지도교수 : 민 경 원
\end{flushright}

%The accurate identification of the dynamic response characteristics of a building structure excited by input signals such as real earthquake or wind load is essential not only for the evaluation of the safety and serviceability of the building structure, but for the verification of an analytical model used in the seismic or wind design. In the field of system identification (SI) which constructs system matrices describing the accurate input/output relationship, it is critical that input should have enough energy to excite fundamental structural modes and a good quality of output containing structural information should be measured.
%In this study, the controller design of an actuator is presented to simulate the responses of a building structure subjected to such dynamic excitations as earthquakes and wind loads. Also, the real-time hybrid shaking table testing method (RHSTTM) is proposed by using only the control device as an experimental part and the load-cell for the feedback reference measuring the interaction force.
\setlength{\baselineskip}{1.5\baselineskip}
지진이나 바람하중과 같은 외부하중으로 인해 가진되는 건축구조물의 동적 응답 특성을 정확하게 식별하는 것은 건축구조물의 안전성과 사용성 평가 뿐만 아니라 내진 또는 내풍 설계에 사용되는 해석 모델의 검증에 필수적이다. 건축구조물의 해석 모델을 구축하기 위하여 정확한 입출력 관계를 설명하는 시스템 행렬을 구성하는 시스템 식별 분야에서는 입력이 주요 모드를 가진시키기에 충분한 에너지를 가져야하고 구조물의 정보를 포함하는 양질의 응답인 출력을 측정하는 것이 매우 중요하다. 그러나 내진 성능 평가를 위하여 실물 크기의 건물을 진동대위에 설치하여 실험하기에 시스템을 구축하기에 매우 어려우며, 내풍 성능 평가를 위하여 실물 크기의 고층 건축물의 실험실에서 풍동 테스트하는 방법은 거의 불가능하다. 따라서 일반적으로 비선형 혹은 비탄성거동이 예측되는 부분구조를 나누어 실험하거나 이를 대체하기 위해 편심가진기나 선형질량가진기등의 가진기를 실물 크기의 사용중 건축물에 설치하고 주요 모드를 가진하여 건축물의 동적 성능 평가를 하거나 시스템 식별을 위한 수단으로 사용된다. 그러나 정확한 내진 및 내풍 성능을 실험적으로 평가하기 위해서 지진하중의 특성과 바람하중의 특성이 모두 충분히 실험에 반영되도록 건축구조물을 가진해야 할 필요가 있다. 

따라서 본 논문은 구조계를 분할하여 실험하는 방식의 하이브리드 실험법과 실물 크기의 건축구조물을 가진하여 동적특성을 식별하기 위한 가진 시스템 설계 기법을 제안한다. 첫째로 하이브리드 실험법은 전체 구조계에서 비선형성이 강한 파트를 실험모델 그리고 선형거동 할 것이라 예측되는 파트를 해석모델로 두고 실시간으로 그 응답을 받아 실험하는 기법이다. 이러한 실험 방법은 실험모델과 해석모델의 인터페이스를 선정해야하며 해석모델의 경계면은 실험모델의 가진력으로 작용하도록 설계하는 실험 시스템을 구축하여야 한다. 본 논문은 진동대를 통해 구조계를 실험구조계와 해석구조계로 분리하여 하이브리드 실험을 하는 부분구조실험법과 비선형 진동제어장치인 동조액체감쇠기, 동조액체기둥감쇠기 그리고 동조액체질량감쇠기를 실험모델로 실험하고 건축구조몰을 해석모델로 계산하는 하이브리드 실험을 설계하고 실험하여 그 성능을 검증하였다. 더나아가 비선형성이 강한 1톤급 층간형 감쇠기인 MR 감쇠기를 UTM에 장착하여 MR 감쇠기가 설치된 건물의 내진성능 평가를 수행하고 MR감쇠기에 준능동 알고리즘을 적용하여 내진 제어성능을 실험적으로 평가하였다. 이는 본 실험 방법을 통해 불특정 비선형성을 가진 제어장치의 최적의 알고리즘 혹은 장치 자체의 성능을 실험적으로 찾아서 적용할 수 있는 가능성과 정량적 근거를 제시할 수 있었다. 두번째로 동적 응답을 구현하는 가진시스템 설계법은 실물크기 사용중 건물에서 선형질량가진기나 능동형질량감쇠기 등을 특정 층에 설치하고 유사지진 또는 유사 바람하중 응답을 구현하는 시스템이다. 본 논문에서는 풍하중 응답 구현을 위해 건물 모델과 풍하중 데이터가 있는 76층 벤치마크 건물을 대상으로 수치해석적으로 검증하였으며, 지진하중 구현을 위해 실물규모의 5층 건물의 HMD를 가진하여 지진하중을 구현하였다. 또한 이를 제어하는 준능동 감쇠장치인 1톤급 MR 감쇠기를 설치하여 실물규모 건물에서 MR감쇠기의 지진에 대한 준능동 제어 성능을 평가하였다. 이러한 기법은 실물크기 건축구조물의 내진 및 내풍성능 평가와 비선형 제어장치가 설치된 건물의 내진 및 내풍성능 또한 평가하는데 사용될 수 있다.
% These techniques can be used to assess the seismic and wind resistance performance of real-scaled building structures and to evaluate the seismic and wind resistance performance of buildings with nonlinear control devices.
% This paper proposes a real-time hybrid test method (RT-HYTEM) and the controller design of actuators to simulate the responses of a building structure subjected to such dynamic excitations as earthquakes and wind loads. Firstly, The real-time hybrid test method is a technique to experiment with a non-linear part in an entire structural system as an experimental model and a part that is expected to behave linearly as an analytical model. In this method, the interface between the experimental model and the analytical model should be selected, and an experimental system should be constructed in which the interface of the analytical model is designed to act as the excitation force of the experimental model. In this paper, a substructure test as part of real-time hybrid test method by separating the structural system into the experimental structure and the analytical structure through the shaking table, and the hybrid test of nonlinear vibration control devices, the tuned liquid damper, the tuned liquid column damper and the tuned liquid mass damper as experimental models and building structures as analytical models were designed and implemented to evaluate the performance. Furthermore, seismic performance evaluation of a building with MR damper installed by installing an MR damper, which is a 1-ton inter-story damper with high nonlinearity, on UTM, and the semi-active algorithm applied to the MR damper were experimentally evaluated. This method can provide the possibility and quantitative basis for experimentally finding and applying the optimum algorithm of the control device with unspecific nonlinearity or the performance of the device itself. Secondary, the controller system design that implements the dynamic response is a system that installs a linear mass shaker or an active mass damper on a specific floor and implements a pseudo-seismic or pseudo wind load response in a full scaled building. In this paper, in order to simulate of wind induced responses, a 76-story benchmark building which has analytical model and wind load data was numerically analyzed, and in order to realize the seismic responses, the pseudo-earthquake excitation is implemented by the hybrid mass damper with real-scaled 5-story building. In addition, an MR damper with the capacity of 1-ton was installed to evaluate the semi-active control performance of MR damper under earthquakes load in a real scale building. These techniques can be used to evaluate the seismic and wind resistance performance of real-scaled building structures and to evaluate the seismic and wind resistance performance of buildings with nonlinear control devices.
% 본 논문의 2장에서 하이브리드 실험법을 제안한다. 첫째로 하이브리드 실험법을 이용한 건축구조물의 부분구조 실험을 진동대를 사용하여 수행하였다. 전단형 건축구조물의 특정 층을 경계면으로 선정하여 목표 구조물의 상부부분을 실험적 부분구조로 그리고 하부부분을 수치해석적 부분구조로 나눈다. 부분구조로 분리된 경계면에 존재하는 경계 하중은 실험 모델의 절대가속도 응답의 실시간으로 피드백하여 계산된다. 두 부분구조의 경계면의 가속도는 수치해석부 부분구조의 실시간 시간이력해석을 통해 계산되고 진동대 제어기의 명령 신호로 사용된다. 이때 명령 신호를 생성하는 진동대 제어기는 명령 신호와 실험부 부분구조의 가진을 위해 생성된 진동대 운동과의 오차를 최소화한다. 하이브리드 실험법의 진동대 부분구조 실험의 유효성과 정확성은 진동대 실험을 통해 실험결과와 수치해석 결과의 일치한 결과를 얻었다. 두 번째로 비선형 제어장치인 동조 액체형 감쇠기(TLD, TLCD 및 TLMD)의 지진가진을 받는 건축구조물의 제어성능 평가를 위한 하이브리드 실험법이 실험적으로 수행되었다. 비선형 제어장치를 실험모델로 건축구조물을 해석모델로 구분하여 실험을 수행하므로 물리적 건축구조물 없이 단지 TLD, TLCD 및 TLMD와 로드 셀 그리고 진동대만을 사용하여 실험한다. TLD, TLCD 그리고 TLMD의 실험체 하부에 장착된 로드 셀을 통해 계측된 전단력과 지진파인 지반가속도를 해석 건물모델에 지반가속도로 가진을 하며 이때 구조물의 응답에서 제어장치가 설치된 층의 절대가속도를 진동대로 구현된다. TLD, TLCD 그리고 TLMD는 설치된 층의 절대가속도로 거동하며 계측된 하부의 전단력을 다시 건물의 설치된 층에 전달하게 되고 설치된 층의 제어 시 혹은 비제어 시 절대가속도를 재생산하며 실험이 진행된다. 본 논문은 하이브리드 실험과 기존의 전체구조계 실험을 비교하여 계측된 응답의 일치한 결과를 얻었고 물리적 구조물 모델 없이 하이브리드 실험법을 사용하여 비선형 진동제어 장치가 설치된 건물의 내진 성능을 정확하게 평가할 수 있음을 제시한다. 본 논문에서 기존 연구와 많은 사례를 통해 검증된 TLD와 TLCD 뿐만 아니라 새롭게 제안한 양방향 응답을 제어하기 위한 장치인 TLMD의 하이브리드 실험법을 통해 양방향 제어성능을 파악함으로써 하이브리드 실험이 설계 단계에서 실험적으로 평가할 수 있어 매우 유용하다.
% Firstly, in Chapter 2, a real-time hybrid test method is proposed. First, a substructuring test of a building structure using the real-time hybrid test method was performed by using a shaking table. Selecting a particular story of a shear-type building structure as an interface, the upper part of the target structure is divided into experimental substructure and the lower part is divided into numerical analytic substructure. The boundary load at the interface separated by the substructure is calculated in real time of the absolute acceleration response of the experimental model. The acceleration of the interface between two substructures is calculated by time history analysis of the numerical substructure and used as command signal of the shaking table controller. At this time, the shaking table controller that generates the command signal minimizes the error between the generated command signal and the shaking table motion for the excitation of the experimental substructure. The validity and accuracy of the shaking table substructure test of real-time hybrid test method were confirmed by the experimental results and the numerical analysis results. Secondly, a real-time hybrid test method for evaluating the control performance of a building structure subjected to earthquake excitation of a tuned liquid type damper (TLD, TLCD, and TLMD), which is a nonlinear control device, was experimentally performed. This experiment is performed by using only control devices (TLD, TLCD, and TLMD), load cell and shaking table without physical building structure because this experiment performs divided into nonlinear control device as an experimental model and building model as an analytical model. The structural responses of the interaction system are calculated numerically in real time using an analytical building model, a given earthquake record, and a shear force generated by the TLD, TLCD, and TLMD, and the shaking table reproduces both the controlled and uncontrolled absolute acceleration of the TLD, TLCD, and TLMD installed floor by modulating the feedback gain of the shear force signal measured by the load cell positioned between the TLD/TLCD/TLMD and the shaking table. In this paper, the results of the real-time hybrid test with those of the existing full-scale structural test are compared, and it is present that accurate results of the seismic performance of buildings with nonlinear vibration control system can be obtained by using real-time hybrid test without a physical model. In this paper, through not only TLD and TLCD which is verified through previous studies but also TLMD which is newly proposed to reduce bidirectional responses of building structures is evaluated by using real-time hybrid test method, a real-time hybrid test method is very useful for that the recently proposed control device can be assessed experimentally in design stage.
% 3장에서는 건축구조물의 동적하중 응답을 구현하기 위한 가진 시스템을 제안한다. 가진기의 힘은 가진기에 의한 구조물의 역전달함수를 통해 계산되고 필터와 포장함수는 가진기의 예측하지 않은 모드 응답과 초기 과도응답을 방지하기 위해 사용되었다. 풍동실험을 통해 얻은 바람 하중이 주어진 76층 벤치마크 건축구조물의 수치해석결과는 특정 층에 설치된 가진 시스템은 각층에 가해지는 바람 하중에 의한 응답을 근사하게 나타낼 수 있음을 보여준다. 제안한 방법에 의해 설계된 가진 시스템은 실제 건축 구조물의 내풍 특성을 평가하는데 그리고 바람 하중 가진을 받는 건축 구조물의 정확한 해석 모델을 수립하는데, 효과적으로 사용될 수 있다. 또한, 복합 질량 감쇠기(Hybrid Mass Damper, HMD)를 이용하여 지진하중을 모사하는 실물 규모의 강제진동 실험이 수행되었다. 구조물의 해석모델은 유한요소법으로 구축하였고, 유한요소모델은 강제진동 실험을 통해 얻은 계측데이터를 통해 주파수 영역과 시간 영역 시스템 식별을 수행하여 수정되었다. 유사지진 가진 실험은 HMD에 의한 층별 가속도 응답이 수정된 유한요소 모델의 지진하중에 의한 응답과 일치함을 보여주었다. 또한, 층간형 제어장치가 설치된 건물의 내진 성능 검증하기 위해 1톤급의 MR 감쇠기를 제작 설치되었으며 지진하중 가진 강제진동실험을 통해 MR 댐퍼의 여러 준능동 알고리즘과 최적 수동 상태에서의 내진 성능을 실험적으로 평가할 수 있었다.
% In Chapter 3, firstly, a design of an actuator for simulating the wind-induced responses of a building structure is presented. The actuator force is calculated by using the inverse transfer function of a target structural responses to the actuator. Filter and envelope function are used such that the error between wind and actuator induced responses is minimized by preventing the actuator from the exciting unexpected modal responses and initial transient responses. The analyses result from a 76-story benchmark building problem in which wind load obtained by wind tunnel test is given, indicate that the actuator installed at a specific floor can approximately embody the structural responses induced by the wind load applied to each floor of the structure. The actuator designed by the proposed method can be effectively used for evaluating the wind response characteristics of a practical building structure and for obtaining an accurate analytical model of the building under wind load. Secondly, a full scale forced vibration test simulating earthquake response is implemented by using a hybrid mass damper. The finite element(FE) model of the structure was analytically constructed using ANSYS and the model was updated using the results experimentally measured by the forced vibration test. Pseudo-earthquake excitation tests showed that HMD (hybrid mass damper) induced floor responses coincided with the earthquake induced ones which were numerically calculated based on the updated FE model. Real-scaled MR damper was installed to verify the effectiveness of the pseudo-earthquake experiment in a real scale building. The MR damper-based control systems are realized when an MR damper is designed by deriving a suboptimal design procedure considering optimization problem and magnetic analysis, and then a damper with the capacity of 1.0 ton is manufactured. In the experiments, a linear active mass driver and the linear shaker seismic simulation testing method are used to excite the building structure to match the full-scale building vibrate as if the building undergoes an earthquake. Under the four historical earthquakes and one filtered artificial earthquake, the performance of the semi-active control algorithms including the passive optimal case is experimentally evaluated. From the experimental results, one can conclude that the Lyapunov and semi-active neuro-control algorithms are appropriate in reducing accelerations of the structural system, and the passive optimal case and the maximum energy dissipation algorithm show the excellent performance in reducing the first-floor displacement.
% 마지막으로 4장에서는 3장에서 실험한 준능동 MR 감쇠기를 이용하여 2장에서 제안한 하이브리드 실험 기법으로 건축구조물에 설치된 MR 감쇠기의 수동 제어와 준능동 제어 성능을 실험적으로 평가하였다. 건물 모델은 3장에서 수행한 실험으로 해석모델을 구축하였으며, MR 감쇠기는 UTM에 장착하여 물리적인 실험모델로 건물 모델을 해석모델로 실험을 수행하게 된다. 먼저 건물 모델의 MR 감쇠기가 위치한 층의 변위가 발생하면 UTM에 그 변위를 전달하게 되며 이때 발생한 MR 감쇠기의 힘은 UTM에 설치된 로드 셀에서 계측된다. 이를 다시 건물 모델을 제어하는 힘으로 환산되어 건물 응답을 제어하는 방식으로 가진 된다. 본 연구에서는 준능동 실험을 위해 하이브리드 실험에서 구조물의 정현파 가진을 통한 공진 시 얻은 실험결과로 MR 감쇠기를 비선형 모델을 단순 Bouc-Wen 모델로 식별하였으며, 매개변수 연구를 수행하였고 이 단순 Bouc-Wen 모델을 이용한 수치해석 결과와 하이브리드 실험 결과와 매우 잘 일치함을 확인하였다. 또한, Passive-on 및 Passive-off 제어에 대한 하이브리드 실험의 결과는 과도한 제어력에 의해 가속도 응답이 많이 감소하지 않았지만, 변위 응답은 MR 댐퍼에 인가된 전류의 증가로 감소하고 있음을 보여주었다. 건물의 최적제어를 위한 MR 댐퍼의 두 개의 준능동 제어알고리즘을 적용하였고 이는 내진 성능과 관련하여 각 알고리즘에 대하여 고찰하였다. 실험과 수치해석의 비교결과에서뿐만 아니라 3장에서 수행한 실물 규모 건물의 지진하중 강제진동 실험 결과와 비교에서도 유사한 결과를 얻어 하이브리드 실험법이 내진 성능 평가 측면에서 기존의 강제진동실험방식과 비교하여 실용적이다.
\\
\\
{\large\textbf{주요어:}}
하이브리드 실험, 부분구조 실험, 비선형 진동제어 장치, 동조 액체 감쇠기, 동조 액체기둥 감쇠기, 동조 액체 질량 감쇠기, 제어-구조 상호력, 가진기 동특성 보상, 능동 질량 감쇠기, 복합형 질량 감쇠기, 가진 제어기 설계, 76층 벤치마크 건축구조물, 강제 진동 실험, 시스템 식별, 유한요소 모델 업데이트, 자기유변유체 감쇠기, 준능동 제어 알고리즘.
\\
\\
학 번 : 72070621\\
\noindent\rule[2pt]{\textwidth}{0.5pt}
