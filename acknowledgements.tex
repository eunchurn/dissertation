\begin{center}
\noindent {\large \textbf{ACKNOWLEDGEMENTS}}\\
\end{center}
% 작은 호기심으로 부터 시작된 저의 학업의 여정은 오랜시간 저에게 도움을 주신 수많은 분들에 의해 박사학위 논문이라는 결실을 맺게 되었습니다. 저에게 박사학위가 가지는 의미는 연구실 그리고 학교의 여러 교수님과 박사님 그리고 선후배의 관계로부터 배우고 훈련된 독립된 연구자라는 의미를 갖습니다. 이제부터는 독립된 연구자로서 어떠한 문제가 주어지더라도(제가 수년간 연구했던 주제 이외에도 다른 문제가 주어진다고 하더라도) 가설을 세우고, 논리적으로 사고하며 수학적인 사고실험 혹은 실제 물리 실험과 데이터를 통하여 통찰하고 문제를 해결하는 능력을 갖추게 되었다고 생각합니다. 저에게 그러한 논리적, 비판적 사고능력, 문제 해결 능력, 그리고 다른 연구자와의 소통능력을 갖도록 도움을 주신 분들에게 이 지면을 빌려 감사의 말씀을 올립니다.
% 우선 저의 지도교수인 민경원 학장님께 감사드립니다. 학부시절 저를 대학원 진학으로 이끄셨고 석사과정 그리고 박사과정 동안 아낌없는 지원과 다양한 연구분야에 관심을 두게 하셨고, 연구자로서 누구나 할 수 있는 일이 아닌 자신만이 할 수 있는 일을 찾도록 그리고 학문에 끊임 없는 열정을 갖도록 저를 훈련시켜 주셨습니다. 또한 학업과 진로에 대해서 관심을 놓지 않으시고 꼼꼼하게 챙겨주심에 감사드립니다. 그리고 저의 논문의 심사위원장이신 이상현 교수님께 감사드립니다. 천재적인 비판적 사고능력과 끊임없는 연구와 학문에 대한 모험 정신과 도전 및 열정을 본받아 배울 수 있었습니다. 그리고 항상 웃음을 잃지 않으시며 저를 대해 주시고 학문 이외의 다양한 분야에서의 철학과 통찰력을 갖도록 해주셨습니다. 그리고 심사위원이신 김준희 교수님께 감사드립니다. 친절하게 저를 맞아 주시고 논문을 꼼꼼하게 지적해 학위논문의 결실을 맺게 해주셨습니다. 또한 전세계적인 학문의 현주소와 수많은 문헌에 대한 통찰을 통해 미래를 대비한 연구자로 나아갈 수 있도록 조언을 해주셨습니다. 그리고 심사위원이신 이성경 교수님께 감사드립니다. 저의 학위논문의 거의 모든 아이디어를 제시해 주시고 제가 실현해 나갈 수 있도록 아낌없는 조언과 지원을 해주셨습니다. 실험을 함께 수행하며 제가 놓칠 수 있었던 세밀한 부분을 모두 챙겨주셨고 논문을 처음부터 끝까지 봐주시며 지원을 해주셨습니다. 그리고 심사위원이신 박지훈 교수님께 감사드립니다. 석박사 과정동안 연구방법론에 대하여 조언을 아끼지 않으셨으며, 문제 해결에 있어 논리적, 비판적 사고능력을 통한 연구방법과 그 결과물을 어떻게 내야하는지 보여주시는 연구자의 롤모델이셨습니다. 현재도 박지훈 교수님의 수많은 연구보고서를 기반으로 학문을 대하는 태도를 본받으려고 노력하고 있습니다.
% 단국대학교 건축공학과에 입학하여 연구하고 박사학위를 받는데 까지 건축공학과의 수많은 교수님들의 도움이 있었습니다. 특히 정란교수님의 학부 명강의에 의해 건축구조공학에 많은 흥미를 가지게 되었고 그 호기심에 의해 대학원을 진학하는 계기가 되었습니다. 또한 대학원 진학후 학생들에 대한 관심과 지지를 아끼지 않으셨고, 실험과 연구에 열정을 가질 수 있도록 도와주셨습니다. 특히 정란 교수님의 구조진동실험실에 대한 지원과 지지가 없었다면 이 학위논문은 완성되지 못했을 것입니다. 또한 저의 석사학위와 박사학위 공개심사에서 아낌없는 조언을 해주셨습니다. 이에 감사드립니다. 그리고 김회서 교수님께 감사드립니다. 학부시절부터 건축 학문에 대한 세계적인 흐름을 바로 읽게 해주시고 이에 대한 학문에 대한 철학과 열정에 대한 가르침을 주셨습니다. 그리고 전재열 교수님께 감사드립니다. 학부시절 강의를 통해 가치공학에 대한 학문의 폭을 넓혀주셨으며, 건축공학 프로젝트에서 엔지니어간의 소통 및 관리 능력의 철학을 배울 수 있었습니다. 학과장이신 김치경 교수님, 그리고 확률에 대한 수학적 지식을 넓혀준 문현준 교수님, 항상 반갑게 맞아주시는 선배님이자 박태원 교수님, 이경구 교수님 그리고 엄태성 교수님께 감사드립니다.
% 그리고 외부에서 본 논문에 많은 아이디어와 조언을 해주신 한양대학교 유은종 교수님, 경북대학교 김홍진 교수님, 여러 학술대회에서 많은 조언을 해주신 KAIST의 윤정방 교수님, 정형조 교수님, 전남대학교 황재승 교수님, 동명대학교 강경수 교수님에게 감사드립니다. 또한 TE Solution의 주석준 박사님, 조승호 박사님, 연구실 선배님이자 저를 많이 독려해 주신 대우건설기술연구소 정진욱 박사님, 연구실 선배님이자 연구실에서 많이 도와주신 현대건설기술연구소 문병욱 박사님, 학과 선배님이자 많은 관심을 주시고 실험을 많이 도와주신 우성식 박사님, Seokjae Heo 그리고 동역학 연구실 김형섭, 이명규, 강상훈, 이루지, 윤경조, 허재성, 정희산, 성지영, 이혜리 이외의 연급되지 않은 모든 연구실 선후배님에게 감사드립니다. 그리고 논문 심사과정에서 많은 도움을 준 신윤수 그리고 행정적인 업무를 맡아 많은 도움을 준 배재윤 조교님에게도 감사드립니다. 현재 근무하고 있는 승화기술정책연구소의 최충기 대표님, 박미연 박사님, 박완순 박사님, 업무가 많이 밀려도 학위 논문을 준비할 수 있도록 배려를 해주셔서 감사드립니다. 그리고 회사에서 저의 잔업을 많이 도와준 문병규, 이정훈 과장에게 감사드립니다. 또한 저를 많이 독려해 주신 한국철도공사 연구원 권세곤 박사님, 박성백 연구원님에게 감사드립니다. 그리고 업무적으로나 연구로나 많은 도움을 주신 강형구 소장님, 김정곤 박사님 그리고 항상 열린 마음으로 저의 이야기를 꼼꼼히 들어주시고 많은 연구 이슈에 대하여 많은 토론을 나누고 발전할 수 있도록 도와주신 Iowa 대학의 이호신 교수님에게 감사의 뜻을 전합니다.
% 연구 이외에 저의 또다른 능력에 관심을 가지고 격려해주신 못의 이이언, 모임별의 조태상(Tae Sang Cho), 조월(Young Sang Cho), 트램폴린의 차효선(Hyo Sun Cha), 오석근(Suk Kuhn Oh) 작가, 캐스커의 이준오(Juno Lee), 저널리스트 문용민(Yongmin Moon, Mimyo), 러브엑스스테레오의 황정익(Toby Hwang), 고연경(Annie Ko), SHEEAN 그리고 Sang Park, 현우현(St. Void), 우연희(Yonhee Wu), 영기획의 하박국 대표, 피터팬컴플렉스의 전지한(Ji-Han Jeon) 그리고 오랜시간 사람12사람으로 함께 해온 지음(배정은)에게 감사드립니다. 또한 수년간 런웨이 작업을 함께하며 희열을 느끼게 해준 HEICH ES HEICH의 한상혁(Sanghyuk Haan) 디자이너, 제 능력을 극대화 시켜준 윤석무 작가님 그리고 조주현(Juhyun Cho), 보드카레인의 주윤하(Yoonha Ju), 홍석우(Sukwoo Hong)님에게 감사드립니다.
% 20년 지기 친구 장인상, 이호상, 이명원

%I have completed the doctoral program at the Dankook University with tremendous assistance from numerous people. There was a lot of help in the process of my getting a doctorate and it would not have been possible without the help of these people. Therefore, I would like to thank them for this article.
%I would like to express deepest appreciation to my advisor, Professor Kyung-Won Min for his friendly support, valuable guidance, endless patience, and constructive supervision through the course of this study. Thanks are also extended to my committee chair Professor Sang-Hyun Lee, who has the attitude and the substance of a genius: he continually and convincingly conveyed a spirit of adventure in regard to research and scholarship, and an excitement in regard to researching. Without his guidance and persistent help this dissertation would not have been possible. I would like to thank my committee members, Professor Junhee Kim whose work demonstrated to me that concern for global affairs supported by an “engagement” in comparative literature and modern technology, should always transcend academia and provide a quest for our times, Professor Ji-Hun Park who did not spare any advice and advice in my research and life during my laboratory time and was a role model about the way of research and how to produce the results. and Professor Sung-Kyung Lee who has presented ideas in almost every area of the study, and has joined us from the beginning of the master's course to the end of the doctoral course. I would like to express deepest appreciation to professors of School of Architecture, during my undergraduate years, Professor Lan Chung's lecture gave me an interest in the structural engineering of architecture, and I was inspired by his passion for academics. In addition, he has participated in both the master's and the doctoral dissertation examination.



My curriculum started with a little curiosity, and the result of my doctoral dissertation was made by a lot of people who helped me for a long time. The meaning of my Ph.D. is that I am an independent researcher trained from the relationship between the professors, the doctors, and the members of the laboratory. Now, as an independent researcher, no matter what problem I am given (even if I have other problems besides the subject I have studied for many years), so I think I have the ability to make a hypothesis, think logically, insight through mathematical thinking experiments or actual physical experiments and data. I am grateful to those who have helped me to have such logical and critical thought ability, problem-solving abilities, and communication skills with other researchers. 

First of all, I would like to thank Professor Kyung-Won Min, my advisor, for his friendly support, valuable guidance, endless patience, and adequate supervision through the course of this study. During my undergraduate years, he led me to graduate school, and he gave me a lot of support and various research fields for my master's and doctoral studies. He made me find out only what I can do for myself as a researcher, not what anyone can do; he has trained me to have a passion that I do not have. Also, I would like to express my gratitude for his paying attention to my studies and my career. I would like to thank my committee chair, Professor Sang-Hyun Lee, who has the attitude and the substance of a genius: he continually and convincingly conveyed a spirit of adventure regarding research and scholarship and excitement regarding researching. I was able to learn from his genius of critical thinking ability, adventurous spirit, challenge, enthusiasm and attitude toward research. And he always laughed and gave me respect and gave me philosophy and insights in various fields other than academic studies. I would like to thank my committee members, Professor Junhee Kim who kindly welcomed me and carefully pointed out my thesis and gave me the result of my dissertation. He also gave insights on the current status of academic studies and numerous works of literature to help us advance toward future researchers. I would like to thank my committee members, Professor Sung-Kyung Lee who gave me almost every idea of my dissertation and gave me generous advice and support so that I could realize it. He carried out the experiment together and took care of all the details that I could miss, and he gave me a look at this thesis from start to finish. I would like to thank my committee members, Professor Ji-Hun Park who was a role model of the researcher who did not spare any advice on the research methodology during the master's course and showed how to research through logical and critical thinking skills and how to deliver the results. Currently, I am trying to get an impression of attitudes toward scholarship based on Professor Park's numerous research reports. 

I was assisted by many professors from the Department of Architectural Engineering until I entered the department of architecture engineering at Dankook University and got my doctorate. In particular, Professor Lan Chung's excellent lectures during my undergraduate years have given me a lot of interest in architecture and structural engineering, and the curiosity from him has made me a graduate student. Also, during my graduate school years, he has been devoted to student interest and support and has helped me to have a passion for experimentation and research. This thesis could not have been completed without his support toward SRRC(Seismic Retrofitting \& Remodeling Research Center) and Structural Analysis \& Dynamics Laboratory. I would like to thank Professor Hway-Suh Kim who has taught me about the philosophy and passion of studying about the global trends in architectural engineering since my undergraduate years. I would like to thank Professor Jae-Yeol Chun. During his undergraduate lectures, I was able to broaden my discipline on value engineering, and I learned the philosophy of communication and management skills among engineers in architectural engineering projects. I also would like to thank Professor Chee-Kyung Kim, who is the head of the department, and Professor Hyeon-Jun Moon, who broadened my mathematical knowledge about probability, Professor Tae-Won Park, Professor Kyung-Koo Lee and Professor Tae-Sung Eom.

Thanks are also extended to Professor Eunjong Yu of Hanyang University, Professor Hongjin Kim of Kyungpook National University, Professor Jae-Seung Hwang of Chonnam National University, Professor Kyung-Soo Kang of Tongmyong University, Dr. Seok-Jun Joo of TE Solution CO., LTD. who gave many ideas and advice to this paper, Dr. Jin-Wook Joung of Daewoo Construction Technology Research Institute, Dr. Byoung-Wook Moon of Hyundai Engineering \& Construction Research Institute who are seniors in the laboratory and encourage me a lot, Dr. Sung-Sik Woo, who is a senior student in the department and has given much attention and helped me a lot with the experiment, Professor Chung-Bang Yun and Professor Hyung-Jo Jung of KAIST who gave a lot of advice in various academic conferences. Thanks are also extended to Hyoung-Seop Kim, Myoung-Kyu Lee, Sang-Hoon Kang, Roo-Jee Lee, Kyung-Jo Youn, Jae-Sung Heo, Hee-San Chung, Ji-Young Seong, Hye-Ri Lee and all alumni of Structural Analysis \& Dynamics Laboratory; I was able to complete my studies with the help of all the graduates. I also thank Yoon-Soo Shin, who helped me a lot in the process of dissertation review, and Jae-Yoon Bae, assistant manager, who helped me with administrative tasks. Mr. Chung-Ki Choi, the representative of S.H. Technology \& Policy Institute, Dr. Mi-Yun Park, Dr. Wan-Soon Park who gave me consideration in preparing dissertation even if there is a lot of work to do, my colleagues, Byung-Gyu Moon and Jeonghoon Lee, who helped me a lot with my work. I would also like to thank Dr. Se-Gon Kwon and Sungbaek Park of Korea Railroad Corporation for encouraging me greatly, Hyung-Goo Kang and Dr. Jung-Gon Kim, who helped me a lot in business and research. I would like to express my sincere appreciation to Professor Hosin David Lee of Iowa University, who has always been open minded to listen to my story and help me to discuss and develop a lot of research issues.

I would like to thank numerous people who encouraged and inspired my other abilities besides research, Yong-Hyun Lee(eAeon) of MOT, Tae-Sang Cho and Young-Sang Cho(Jowall) of Byul.org, Hyo-Sun Cha of Trampauline, Photographer Suk-Kuhn Oh, Juno Lee of Casker, Journalist Yongmin Moon(mimyo), Toby Hwang and Annie Ko of LOVEXSTEREO, SHEEAN and Sang Park, Wuhyun Hyun(St. Void) and Yonhee Wu, Min-Hun Yoon(HAVAQQUQ) of Young, Gifted\&Wack, Ji-Han Jeon of Peterpan Complex and Jeong-Eun Bae(Zieum) who has waited for me to make music for a long time as a team of SARAM12SARAM. Thanks are also extended to Sang-Hyuk Haan, a designer, CEO and founder of HEICH ES HEICH who has been working together on the runway for many years and made me feel alive, Photographer Sukmu Yoon, Juhyun Cho, Yoonha Ju of Vodka Rain who maximized my potential. I would like to thank In-Sang Chang, Ho-Sang Lee, Myoung-Won Lee for being an old friend over 20 years. I am grateful to everyone who knows me. Finally, I wish to express my love and gratitude for my wife Inkyung Baik, and wife's family. Also, I am greatly indebted to my parents, my brother for their love and support.
\\
\hspace*{\fill} July 2017\\
\hspace*{\fill} Eunchurn Park