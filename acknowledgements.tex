\begin{center}
\noindent {\large \textbf{Acknowledgements}}\\
\end{center}
작은 호기심으로 부터 시작된 저의 학업의 여정은 오랜시간 저에게 도움을 주신 수많은 분들에 의해 박사학위 논문이라는 결실을 맺게 되었습니다. 저에게 박사학위가 가지는 의미는 연구실 그리고 학교의 여러 교수님과 박사님 그리고 선후배의 관계로부터 배우고 훈련된 독립된 연구자라는 의미를 갖습니다. 이제부터는 독립된 연구자로서 어떠한 문제가 주어지더라도(제가 수년간 연구했던 주제 이외에도 다른 문제가 주어진다고 하더라도) 가설을 세우고, 논리적으로 사고하며 수학적인 사고실험 혹은 실제 물리 실험과 데이터를 통하여 통찰하고 문제를 해결하는 능력을 갖추게 되었다고 생각합니다. 저에게 그러한 논리적, 비판적 사고능력, 문제 해결 능력, 그리고 다른 연구자와의 소통능력을 갖도록 도움을 주신 분들에게 이 지면을 빌려 감사의 말씀을 올립니다.
우선 저의 지도교수인 민경원 학장님께 감사드립니다. 학부시절 저를 대학원 진학으로 이끄셨고 석사과정 그리고 박사과정 동안 아낌없는 지원과 다양한 연구분야에 관심을 두게 하셨고, 연구자로서 누구나 할 수 있는 일이 아닌 자신만이 할 수 있는 일을 찾도록 그리고 학문에 끊임 없는 열정을 갖도록 저를 훈련시켜 주셨습니다. 또한 학업과 진로에 대해서 관심을 놓지 않으시고 꼼꼼하게 챙겨주심에 감사드립니다. 그리고 저의 논문의 심사위원장이신 이상현 교수님께 감사드립니다. 천재적인 비판적 사고능력과 끊임없는 연구와 학문에 대한 모험 정신과 도전 및 열정을 본받아 배울 수 있었습니다. 그리고 항상 웃음을 잃지 않으시며 저를 대해 주시고 학문 이외의 다양한 분야에서의 철학과 통찰력을 갖도록 해주셨습니다. 그리고 심사위원이신 김준희 교수님께 감사드립니다. 친절하게 저를 맞아 주시고 논문을 꼼꼼하게 지적해 학위논문의 결실을 맺게 해주셨습니다. 또한 전세계적인 학문의 현주소와 수많은 문헌에 대한 통찰을 통해 미래를 대비한 연구자로 나아갈 수 있도록 조언을 해주셨습니다. 그리고 심사위원이신 이성경 교수님께 감사드립니다. 저의 학위논문의 거의 모든 아이디어를 제시해 주시고 제가 실현해 나갈 수 있도록 아낌없는 조언과 지원을 해주셨습니다. 실험을 함께 수행하며 제가 놓칠 수 있었던 세밀한 부분을 모두 챙겨주셨고 논문을 처음부터 끝까지 봐주시며 지원을 해주셨습니다. 그리고 심사위원이신 박지훈 교수님께 감사드립니다. 석박사 과정동안 연구실에서 함께해주시며 연구방법론에 대하여 조언을 아끼지 않으셨으며, 문제 해결에 있어 논리적, 비판적 사고능력을 통한 연구방법과 그 결과물을 어떻게 내야하는지 보여주시는 연구자의 롤모델이셨습니다. 현재도 박지훈 교수님의 수많은 연구보고서를 기반으로 학문을 대하는 태도를 본받으려고 노력하고 있습니다.
단국대학교 건축공학과에 입학하여 연구하고 박사학위를 받는데 까지 건축공학과의 수많은 교수님들의 도움이 있었습니다. 특히 정란교수님의 학부 명강의에 의해 건축구조공학에 많은 흥미를 가지게 되었고 그 호기심에 의해 대학원을 진학하는 계기가 되었습니다. 또한 대학원 진학후 학생들에 대한 관심과 지지를 아끼지 않으셨고, 실험과 연구에 열정을 가질 수 있도록 도와주셨습니다. 특히 정란 교수님의 구조진동실험실에 대한 지원과 지지가 없었다면 이 학위논문은 완성되지 못했을 것입니다. 또한 저의 석사학위와 박사학위 공개심사에서 아낌없는 조언을 해주셨습니다. 이에 감사드립니다. 그리고 김회서 교수님께 감사드립니다. 학부시절부터 건축 학문에 대한 세계적인 흐름을 바로 읽게 해주시고 이에 대한 학문에 대한 철학과 열정에 대한 가르침을 주셨습니다. 그리고 전재열 교수님께 감사드립니다. 학부시절 강의를 통해 가치공학에 대한 학문의 폭을 넓혀주셨으며, 건축공학 전반에 대하여 
%I have completed the doctoral program at the Dankook University with tremendous assistance from numerous people. There was a lot of help in the process of my getting a doctorate and it would not have been possible without the help of these people. Therefore, I would like to thank them for this article.
%I would like to express deepest appreciation to my advisor, Professor Kyung-Won Min for his friendly support, valuable guidance, endless patience, and constructive supervision through the course of this study. Thanks are also extended to my committee chair Professor Sang-Hyun Lee, who has the attitude and the substance of a genius: he continually and convincingly conveyed a spirit of adventure in regard to research and scholarship, and an excitement in regard to researching. Without his guidance and persistent help this dissertation would not have been possible. I would like to thank my committee members, Professor Junhee Kim whose work demonstrated to me that concern for global affairs supported by an “engagement” in comparative literature and modern technology, should always transcend academia and provide a quest for our times, Professor Ji-Hun Park who did not spare any advice and advice in my research and life during my laboratory time and was a role model about the way of research and how to produce the results. and Professor Sung-Kyung Lee who has presented ideas in almost every area of the study, and has joined us from the beginning of the master's course to the end of the doctoral course. I would like to express deepest appreciation to professors of School of Architecture, during my undergraduate years, Professor Lan Chung's lecture gave me an interest in the structural engineering of architecture, and I was inspired by his passion for academics. In addition, he has participated in both the master's and the doctoral dissertation examination.



