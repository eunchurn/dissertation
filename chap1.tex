%!TEX encoding = UTF-8 Unicode
\chapter{Introduction}
\label{chap:intro}
\section{General}
Accurate identification of the dynamic response characteristics of building structures excited by external loads such as earthquakes and wind loads is essential not only for the safety and serviceability evaluation of building structures but for the validation of analytical models used for seismic or wind design\citep{ljung1987system}. In the System Identification (SI) field, which constructs a system matrix that describes precise input and output relationships to build an analytical model of an architectural structure, it is critical that input should have sufficient energy to excite fundamental structural modes and a high quality of output containing structural information should be measured\citep{alvin1994second,madenci1994free}. However, full-scale dynamic testing is difficult due to size and weight limitations in most large civil engineering structures. Also, in most cases, it is very difficult or impossible to extrapolate the results of small-scale model tests as a result of a full-scale structure that operates non-linearly in most cases. Therefore, it is used as a means to evaluate the dynamic performance of a building or to implement the system identification process by installing an excitation system such as an eccentric exciter or a linear mass shaker which excite the fundamental modes, in the building. However, to evaluate the seismic and the wind resist performance accurately, it is necessary to fully reflect the characteristics of the seismic load and the wind load.

One method to overcome these limitations is to use the hybrid testing method. This technique for the dynamic testing of large-scale structures, typically involving nonlinearities, does not test the entire structure as a whole. Instead, it often happens that nonlinearity or unpredictable behavior is concentrated in a limited number of local parts of the overall structure. Those important local parts are experimentally tested, while the rest of the structure, which is assumed to behave linearly, in essence, is modeled numerically. Numerical and experimental parts are allowed to interact with each other to produce the response of the entire system. The advantage of the hybrid testing technique over the reduced-scale model testing is that the experimental substructure is tested at full size and the scaling effect can be eliminated. 

Another method to overcome above limitations is to design controller of actuators to simulate earthquake and wind-induced responses. This is a forced vibration testing method to implement seismic or wind responses of a building structure by using a vibration control device already installed in a real building such as ATMD or installing additional exciter in specific story of the building. For the successful application of this method, several issues should be considered. First, interaction effects between the actuators and building structure and dynamic characteristics of the actuator itself should be compensated. Second, because this method uses inverse transfer function method, unexpected modal responses and transient responses should be reduced. This method may evaluate the seismic and wind performance not only of the existing building but for of the real scale building equipped with the nonlinear vibration control device.


\section{Objectives and Contents}
Figure~\ref{fig:subject} shows the contents of chapters and their relations in this paper. It can be classified into a hybrid testing method and a pseudo-excitation system for simulating a dynamic load of building structures. Verification methods can be summarized as numerical analysis, shaking table experiment, UTM experiment, and experiment with full-size building with HMD. The objective of this study is to design a hybrid testing method and a controller of an exciter to perform an experimental evaluation of seismic and wind resistance performance of building structures equipped with nonlinear vibration control devices. The contents for these objectives are as follows;

First, in chapter~\ref{rthytem}, the hybrid testing method is proposed. To be implemented by the hybrid testing technique and accurately represent the whole building structure, the experimental and numerical substructures must operate in parallel with minimal errors at the interface between them. Three factors are essential in the implementation issues of the hybrid testing method. Those issues are the measurement of the interface force between two substructures, the calculation of the numerical substructure, and the loading operation of the experimental substructure with an actuator. The responses of the shared degrees-of-freedom (DOF) are calculated using the numerical model and passed to the controller of actuators or a shaking table. The controller generates signals driving the actuators in order to impose these responses on the experimental substructure. To be replicated by the behavior of the whole structure by the hybrid testing method, it is important to design the controller of the actuators or the shaking table, so that difference between the expected and actual measured behavior at the interface of the two substructures is minimized. The vibration control effect of a tuned liquid type damper (TLD/TLCD) for a building structure excited by earthquake load is experimentally evaluated through the hybrid testing method. The hybrid testing method doesn’t require a physical building structural model in performing the experiment of a TLD/TLCD-structure interaction system and it only uses a TLD/TLCD which is known to have strong nonlinearity dependent on response amplitude, excitation frequency, and depth to length ratio \citep{yalla2001liquid}. A new control device, which is called tuned liquid mass damper (TLMD), is discussed in this chapter. The functional characteristic of a TLMD used in this study is that its mass is composed of both a mass of TLCD tank itself and that of liquid within a tank. Natural rubbers were also used to substitute the stiffness of a TMD. Therefore, a TLMD employed in this study operates as a TLCD in one direction and behaves as a TMD in the other orthogonal direction. Regarding the field of structural testing, hybrid testing method also implemented in this study. As mentioned in the previous sentence, TLD and TLCD has been subjected to many researches and experiments and has been verified by the hybrid testing method in this study. Therefore, as hybrid testing method of building controlled by TLMD, it is significant to use the hybrid testing method technique by validating the control performance of the newly proposed control device by comparison between the SDOF-actuator experiment method and the hybrid testing method method. 
The structural responses of the interaction system are calculated numerically in real-time using the analytical structural model with the excitations of measured control force, user-defined base earthquake loads, and its state space realization incorporated in the integrated controller of an actuator. Also, in order to minimize the distortion of the acceleration of the shaking table or an actuator's displacement, the inverse transfer function of the shaking table or the actuator is identified and its state space realization is implemented in the hybrid testing method controller.

Secondly, in chapter~\ref{chap:6}, simulation of dynamic responses of a building structure using a linear mass shaker is conducted. In order for the linear shaker to keep the structure in the target response trajectory, an inverse transfer function of a structural response to the shaker force is obtained using a state space form governing equation of the structure and the discrete Fourier transform of structural response is performed. Filter and envelope function are used such that the error between wind and actuator induced responses is minimized by preventing the shaker from the exciting unexpected modal response and initial transient response. The effectiveness of the proposed method is verified through a numerical example of a 76 story-benchmark building excited by wind load of which deterministic time history is given. The effect of the type of the target structural response on a convergence of the shaker force signal and the error magnitude is investigated. Moreover, in this chapter, forced vibration pseudo-earthquake tests using an HMD(hybrid mass damper) installed on the 4th floor of the real-scale five-story modal testing tower were performed. To experimentally evaluate the seismic performance of a building structure with accuracy, shaking table test should be conducted. However, it is practically impossible and very expensive to excite a real scale structure using the shaking table. Yu presented a linear shaker system to simulate structural seismic response by using inverse transfer function of the structural response induced by the linear shaker \citep{yu2005forced} is of the structure excited by earthquake load. Real-scaled MR damper was installed to verify the effectiveness of the pseudo-earthquake experiment in a real scale building. Under the four historical earthquakes and one filtered artificial earthquake, the performance of the semi-active control algorithms including the passive optimal case is experimentally evaluated by the pseudo-earthquake test method.



Finally, in chapter~\ref{chap:rthytem-mrdamper} shows the experimental verification of hybrid testing method applied to UTM with an MR damper. The MR damper with the capacity of 1-ton was used from chapter~\ref{chap:6}, hybrid testing method of the real-scaled building with semi-active controlling MR damper implemented in the chapter~\ref{rthytem} was applied by hybrid testing method technique. This study presents the quantitative evaluation of the seismic performance of a building structure installed with a MR damper using hybrid testing method. A building model is identified from the forced vibration testing results of a full-scale five-story building and is used as the numerical substructure, and an MR damper corresponding to an experimental substructure is physically tested using a universal testing machine (UTM). First, the force required to drive the displacement of the story, at which the MR damper is located, is measured from the load cell attached to the UTM. The measured force is then returned to a control computer to calculate the response of the numerical substructure. Finally, the experimental substructure is excited by the UTM with the calculated response of the numerical substructure. The hybrid testing method implemented in this study is validated because the hybrid testing results obtained by application of sinusoidal and earthquake excitations and the corresponding analytical results obtained using the Bouc-Wen model as the control force of the MR damper with input currents are in good agreement. Also, the results from hybrid testing method for the passive -on and -off control show that the structural responses did not decrease further by the excessive control force, but decreased due to the increase of the current applied to the MR damper. Also, two semi-active control algorithms (modulated homogeneous friction and the clipped-optimal control algorithms) are applied to the MR damper in order to optimally control the structural responses. To be compared with the hybrid testing method and numerical results, the Bouc-Wen model parameters should be identified for each input current. The results of the comparison of experimental and numerical responses show that it is more practical to use hybrid testing method in semi-active devices such as MR dampers. The test results indicate that a control algorithm can be experimentally applied to the MR damper using hybrid testing method. This study also provides a discussion on each algorithm with respect to the seismic performances.

\begin{figure}[ht]
\centering
\includegraphics[width=1\textwidth] {figure/contents.eps}
\label{fig:subject}
\caption{Contents of chapters and theier relations}
\end{figure}

\section{Literature Review}
\subsection{Hybrid testing method}
Since the substructuring technique is first developed for large-scale structures by \citet{nakashima1992development}, several pieces of research have been performed on the substructuring technique both experimentally and analytically to overcome difficulties in the test of large-scale structures. \citet{blakeborough2001development,darby2001improved} focused on the control algorithm for operating hydraulic actuators and conducted experiments for a one-story frame with horizontal and rotational DOF and its corresponding numerical substructure, which shows the linear or nonlinear behavior \citep{darby2002stability}. \citet{neild2005control} separated a large structural mass of the single DOF system into two parts and selected the smaller one as the experimental substructure and the larger one with attached spring and dashpot as the numerical substructure to conduct a shaking table test\citep{neild2005control}. Nakashima and Pan applied the method to a base-isolated structure consisting of the base-isolation layer as an experimental substructure and the rest of the upper structural model as a numerical substructure\citep{nakashima1999real}. Besides, they developed a mixed control algorithm using both displacement and force for its implementation\citep{pan2005online}.

\citet{takanashi1975nonlinear,takanashi1980inelastic} have firstly developed the pseudo-dynamic testing method, in which only a part of the whole structure, mainly being expected to show nonlinearity, is manufactured and tested while the remainder showing linearity is numerically calculated\citep{takanashi1975nonlinear,takanashi1980inelastic}. Because there exists propagation of experimental errors in pseudo-dynamic testing methods, the stability, and accuracy of the numerical integration methods were investigated\citep{shing1983experimental,shing1984pseudodynamic}. In these pseudo-dynamic testing methods, as implied in the name, the experimental part is not \textbf{``dynamically''} but \textbf{``statically''} excited under the loading condition which makes the testing part represent identical displacement response to that of the part in whole structure excited by considered dynamic load such as ground acceleration.

\citet{horiuchi1999development} compensated the time-delay effect caused mainly by analytical procedure in the RHTM\citep{horiuchi1999development}. For its experimental verification, a small portion of mass was separated from a mass-spring-dashpot system, and only the small part of the mass was tested considering the effects of the other parts analytically. \citet{iemura1999substructured,igarashi2000development} verified the effect of vibration control devices such as a tuned mass damper and an active mass damper installed in a structure excited by ground acceleration, using the hybrid testing method in which the control devices were experimental parts and the remaining structural model was a numerical part. The acceleration signal of the moving mass of the devices was measured and used as input to the numerical model\cite{iemura1999substructured,igarashi2000development}.

TLD dissipates structural vibratory energy by tuning the frequency of the liquid sloshing to one of the structure\citep{soong1997passive}. TLD has been applied to the control of wind-induced acceleration response\citep{chang1998unified}, and recently, some investigations on the seismic control performance of the TLD have been made\citep{banerji2000tuned}. In order to describe the behavior of the TLD, linear model based on tuned mass damper analogy\citep{sun1995properties} and linear wave theory, nonlinear stiffness and damping model\citep{yu1999non}, and sloshing-slamming analogy\citep{yalla2001liquid} can be used. However, because any model has the error in capturing the real dynamic characteristics of the control force generated by the TLD and furthermore the error increases for the case of non-stationary excitation such as the earthquake, evaluation of the seismic control performance of the TLD only numerically has the accuracy problem.

Recently, the TLCD has received the attention of researchers as a type of auxiliary mass system\citep{samali1998wind}. TLCD has the control characteristics similar to that of tuned mass damper (TMD), which is one of most frequently used dampers for vibration control. Since the viscosity term in the governing equation of motion of TLCD is a function of the absolute value of liquid velocity, the equation is non-linear, and the dynamic characteristics of TLCD depend on the magnitude and the characteristics of excitation forces and the corresponding structural responses of the floor at which TLCD is installed\citep{yalla2001liquid}.

Two or more TMDs and TLCDs have been installed in a building to control bidirectional responses\citep{xu1992dynamic, fujino1993vibration, yamaguchi1993fundamental, igusa1994vibration, jangid1997performance, li2000performance, li2000optimum, singh2002tuned} . Since multi-tuned mass dampers (MTLDs) had been proposed by \citet{xu1992dynamic,igusa1994vibration}, many researchers have devoted their efforts to improve the control performance of MTLDs\citep{fujino1993vibration, yamaguchi1993fundamental, jangid1997performance, li2000performance, li2000optimum, singh2002tuned, zhang2004equivalent}. Especially, \citet{jangid1997performance, singh2002tuned} have studied the MTMDs by which the torsional response of building structures subjected to bidirectional vibrations can be reduced. They also proposed the optimum parameters such as mass, frequency and damping ratios, in addition to the installation locations of MTMDs, to control the torsional response. \citet{fujino1993vibration} have studied about the optimal frequency ratio and the optimal number of MTLDs. However, for applying these multi dampers to building structures, there have been shortcomings such as high installation cost, hard maintenance and additional retrofitting of the story at which they are installed. Additionally, applying the pendulum type TMD used in Taipei 101 building has a limitation that the structural plan should be the similar shape in both weak and strong axes. Also, this type of TMD needs a large installation space\citep{haskett2004tuned}.

\subsection{Design of Excitation System for Simulating Dynamic Loads}

In the field of earthquake engineering, forced vibration tests for evaluating the dynamic characteristics of structures and evaluating seismic performance have been widely carried out from shaking table, eccentric exciter, and reaction wall to shrinking and real scale structures.
Several studies on the system identification to establish the analytical structural model by using the modal characteristics, which are obtained by forced vibration tests, have also been carried out\citep{halling2001dynamic,min2004vibration}.
The primary purpose of system identification is to construct an analytical model that can simulate the relationship between the excitation load and the measurement response obtained in the forced excitation experiment.

\citet{juang1994applied} proposed the system identification method in the frequency domain by using matrix decomposition and that in the time domain by employing Markov parameters and Observer/Kalman filter\citep{juang1994applied}. \citet{dyke1994experimental} obtained the system matrix of a structure by applying model reduction technique after exciting a 3-story small scale structural model with the shaking table and the active mass driver, and then performing the system identification for the respective excitation loads. However, the analytical model that is identified by adopting their methods only describes its input and output relations and does not represent physical information such as the mass, viscosity, and stiffness\citep{dyke1994experimental}. \citet{alvin1994second} extracted the finite element model that has its physical meaning by using common based-normalized system identification (CBSI) technique\citep{alvin1994second}.

The shaking table test, in which the loads similar to real earthquakes can be easily simulated, is frequently taken to experimentally assess the seismic performance of a structure with high accuracy. It is usually applied to tests on small-scale structure, however, since it is very difficult to excite real-scaled structures with a shaking table. \citet{yu2005forced} designed the linear shaker system to simulate earthquake response of a structure by employing the methods both minimizing the error in time histories between the earthquake-induced structural responses and the linear shaker-induced ones and using the inverse transfer function of the responses of a structure loaded by a linear shaker\citep{yu2005forced}. However, their study is based on numerical verification, and also has several limitations for being applied to real-scaled structures. First, forced vibration tests are not appropriate for utilizing the vibration source with large amplitude in the experimental structure. Especially, the structure can not be excited to its nonlinear range, since safety may not be ensured during the test and the linear structural model is used for the numerical simulation. Secondly, it is inappropriate for using the vibration source with full bandwidth. Thirdly, sensors may not be sufficiently deployed in field test due to expenses and their installations so that the structural members, in which critical behaviors are expected, are detected. Finally, the process of obtaining the system matrix of a structure by using the system identification technique should be additionally included in the entire research work, since the structure is based on an analytical model. Accordingly, the validity of the proposed pseudo-earthquake excitation test needs to be experimentally investigated under these constraints. 

\citet{dyke1994experimental} obtained controller canonical form state space realization for a small scale three-story building by using both active mass driver and shaking table and measuring the absolute floor acceleration. \citet{juang1994applied} proposed Observer/Kalman filter identification using system Markov parameters in time-domain. These studies present the mathematical models which accurately describe input/output relationship but do not provide physical mass, stiffness, and damping matrices, finite element model based SI techniques are developed in the field of health monitoring or damage detection\citep{van2003sensors}. \citet{yu2005forced} performed a series of ambient vibration measurement and force vibration tests on four-story reinforced concrete building by using linear and eccentric mass shakers and updated the analytical finite element model based on the collected dynamic data. Also, \citet{yu2005forced} presented a linear shaker system to simulate linear elastic structural seismic response. However, because it is practically very difficult to excite large-scale civil structures by using artificial actuator-type devices, SI techniques using natural excitation such as the wind, vehicle, and human are investigated.
In the most earthquake engineering, forced vibration testing has been implemented for many years to assess dynamic characteristics of a building structure. It is based on the fact that dynamic response is sensitive to changes in mass, damping, or stiffness of a structure. These changes would lead to shifts in the modal parameters, such as natural frequencies and mode shape. The dynamic characteristics in the system to identify the integrity of a structure have also been investigated\citep{dyke1994experimental}. The primary objective in system identification is to find the relationship between the experimental response data and the analytical model by adjusting the structural parameters in the model.


\subsection{Hybrid Testing Method on a Semi-actively Controlled Building Structure Equipped with Full-scale MR Dampers}

A wide variety of structural control strategies for civil engineering structures such as bridges and buildings have been proposed by structural engineers\citep{adeli1999control}. Various types of active and semi-active or hybrid control dampers have also been developed to enhance the performance of passive control devices\citep{kim2005hybrid1,kim2005hybrid2}. Because semiactive control devices such as variable-orifice dampers, variable-friction dampers, and electrorheological/magnetorheological (ER/MR) dampers have the potential to offer the adaptability of active devices without requiring the associated large power sources as well as the inherent stability, it is expected that the application of a semiactive control device is one of the most promising means for mitigating structural response of building structures under the unexpected external loads such as earthquake, wind, blast, impact, and so on. 

Analytical and experimental studies on these control devices when applied to MR dampers have been performed in an effort to reduce the structural responses, mainly due to their intrinsic stability and low power consumption(e.g., high dynamic range, mechanical simplicity, environmental robustness, etc.) since mid-1990s \citep{dyke1994experimental, spencer1997phenomenological}. MR damper-based semiactive control systems require a feedback system consisting of sensors, external power supply and controllers. Semi-active and adaptive control strategies using the MR damper have also been extensively compared and analyzed\citep{jansen2000semiactive,kim2009semiactive,bitaraf2010adaptive}. Also a lot of control algorithms for MR damper-based semiactive systems have been proposed and numerically verified for control of civil engineering structures such as buildings and bridges by several researchers \citep{jansen2000semiactive,zhou2003adaptive}. However, each has emphasized its advantages depending on the specific application and desired effect. In order to demonstrate the pros and cons of the different control algorithms, a series of benchmark structural control problems have been developed by the ASCE Structural Control Committee and Task Group on Benchmark Problems and International Association of Structural Control and Monitoring (IASCM). Although the benchmark problems may give us a chance to compare various control algorithms directly, they are only numerical simulations which cannot consider practical limitations. In analytical studies, the Bingham, Bouc-Wen, and phenomenological models were proposed as analytical models for describing the hysteretic behavior of the MR damper\citep{spencer1997phenomenological,yang2004dynamic}. Although these models are useful in the design of the MR damper, they are inappropriate for characterizing the behavior of the MR damper under the excitation of irregular loads such as earthquakes and winds because of the high nonlinearities of the MR damper due to its dependency on the loading rate and the amplitude of excitations. Also, the performance of the MR damper is not guaranteed in accordance with its current providing devices. Moreover, when the MR damper behaves as a semiactive control device, the hysteretic model is unreliable when it varies with the applied current\citep{lee2006semi}. For these reasons, there may be disagreement between the corresponding analytical results and the actual responses of a building installed with an MR damper when the semi-active control strategy is applied.