\begin{center}
\noindent {\large \textbf{Hybrid Testing Method and Excitation System for Seismic and Wind-resistance Performance Evaluation of Building Structures with Nonlinear Dampers}}
\end{center}
\begin{flushright}
Park, Eunchurn\\
Department of Architectural\\
Engineering Graduate School\\
Dankook University\\
\vspace{0.3cm}
Advisor: Professor Min, Kyung-Won
\end{flushright}

{\large\textbf{Abstract:}}
% The accurate identification of the dynamic response characteristics of a building structure excited by input signals such as real earthquake or wind load is essential not only for the evaluation of the safety and serviceability of the building structure, but for the verification of an analytical model used in the seismic or wind design. In the field of system identification (SI) which constructs system matrices describing the accurate input/output relationship, it is critical that input should have enough energy to excite fundamental structural modes and a good quality of output containing structural information should be measured.
This paper proposes a real-time hybrid testing method and the controller design of actuators to simulate the responses of a building structure subjected to such dynamic excitations as earthquakes and wind loads. Firstly, The real-time hybrid testing method is a technique to experiment with a non-linear part in an entire structural system as an experimental model and a part that is expected to behave linearly as an analytical model. In this method, the interface between the experimental model and the analytical model should be selected, and an experimental system should be constructed in which the interface of the analytical model is designed to act as the excitation force of the experimental model. In this paper, a substructure test as part of real-time hybrid testing method by separating the structural system into the experimental structure and the analytical structure through the shaking table, and the hybrid test of nonlinear vibration control devices, the tuned liquid damper, the tuned liquid column damper and the tuned liquid mass damper as experimental models and building structures as analytical models were designed and implemented to evaluate the performance. Furthermore, seismic performance evaluation of a building with magneto-rheological (MR) damper installed by installing an MR damper, which is a 1-ton inter-story damper with high nonlinearity, on UTM, and the semi-active algorithm applied to the MR damper were experimentally evaluated. This method can provide the possibility and quantitative basis for experimentally finding and applying the optimum algorithm of the control device with unspecific nonlinearity or the performance of the device itself. Secondary, the controller system design that implements the dynamic response is a system that installs a linear mass shaker or an active mass damper on a specific floor and implements a pseudo-seismic or pseudo wind load response in a full-scale building. In this paper, in order to simulate of wind-induced responses, a 76-story benchmark building which has an analytical model and wind load data were numerically analyzed, and in order to realize the seismic responses, the pseudo-earthquake excitation is implemented by the hybrid mass damper with the real-scaled 5-story building. In addition, an MR damper with the capacity of 1-ton was installed to evaluate the semi-active control performance of MR damper under earthquakes load in a real scale building. These techniques can be used to evaluate the seismic and wind resistance performance of real-scaled building structures and to evaluate the seismic and wind resistance performance of buildings with nonlinear control devices.

Firstly, in Chapter 2, a real-time hybrid testing method is proposed. First, a substructuring test of a building structure using the real-time hybrid testing method was performed by using a shaking table. Selecting a particular story of a shear-type building structure as an interface, the upper part of the target structure is divided into experimental substructure and the lower part is divided into numerical analytic substructure. The boundary load at the interface separated by the substructure is calculated in real time of the absolute acceleration response of the experimental model. The acceleration of the interface between two substructures is calculated by time history analysis of the numerical substructure and used as command signal of the shaking table controller. At this time, the shaking table controller that generates the command signal minimizes the error between the generated command signal and the shaking table motion for the excitation of the experimental substructure. The validity and accuracy of the shaking table substructure test of real-time hybrid testing method were confirmed by the experimental results and the numerical analysis results. Secondly, a real-time hybrid testing method for evaluating the control performance of a building structure subjected to earthquake excitation of a tuned liquid type damper (TLD, TLCD, and TLMD), which is a nonlinear control device, was experimentally performed. This experiment is performed by using only control devices (TLD, TLCD, and TLMD), load cell and shaking table without physical building structure because this experiment performs divided into nonlinear control device as an experimental model and building model as an analytical model. The structural responses of the interaction system are calculated numerically in real time using an analytical building model, a given earthquake record, and a shear force generated by the TLD, TLCD, and TLMD, and the shaking table reproduces both the controlled and uncontrolled absolute acceleration of the TLD, TLCD, and TLMD installed floor by modulating the feedback gain of the shear force signal measured by the load cell positioned between the TLD/TLCD/TLMD and the shaking table. In this paper, the results of the real-time hybrid test with those of the existing full-scale structural test are compared, and it is present that accurate results of the seismic performance of buildings with nonlinear vibration control system can be obtained by using real-time hybrid test without a physical model. In this paper, through not only TLD and TLCD which is verified through previous studies but also TLMD which is newly proposed to reduce bidirectional responses of building structures is evaluated by using real-time hybrid testing method, a real-time hybrid testing method is very useful for that the recently proposed control device can be assessed experimentally in design stage.

In Chapter 3, firstly, a design of an actuator for simulating the wind-induced responses of a building structure is presented. The actuator force is calculated by using the inverse transfer function of a target structural responses to the actuator. Filter and envelope function are used such that the error between wind and actuator induced responses is minimized by preventing the actuator from the exciting unexpected modal responses and initial transient responses. The analyses result from a 76-story benchmark building problem in which wind load obtained by wind tunnel test is given, indicate that the actuator installed at a specific floor can approximately embody the structural responses induced by the wind load applied to each floor of the structure. The actuator designed by the proposed method can be effectively used for evaluating the wind response characteristics of a practical building structure and for obtaining an accurate analytical model of the building under wind load. Secondly, a full scale forced vibration test simulating earthquake response is implemented by using a hybrid mass damper. The finite element(FE) model of the structure was analytically constructed using ANSYS and the model was updated using the results experimentally measured by the forced vibration test. Pseudo-earthquake excitation tests showed that HMD (hybrid mass damper) induced floor responses coincided with the earthquake induced ones which were numerically calculated based on the updated FE model. Real-scaled MR damper was installed to verify the effectiveness of the pseudo-earthquake experiment in a real scale building. The MR damper-based control systems are realized when an MR damper is designed by deriving a suboptimal design procedure considering optimization problem and magnetic analysis, and then a damper with the capacity of 1.0 ton is manufactured. In the experiments, a linear active mass driver and the linear shaker seismic simulation testing method are used to excite the building structure to match the full-scale building vibrate as if the building undergoes an earthquake. Under the four historical earthquakes and one filtered artificial earthquake, the performance of the semi-active control algorithms including the passive optimal case is experimentally evaluated. From the experimental results, one can conclude that the Lyapunov and semi-active neuro-control algorithms are appropriate in reducing accelerations of the structural system, and the passive optimal case and the maximum energy dissipation algorithm show the excellent performance in reducing the first-floor displacement.

In chapter 4, The MR damper with the capacity of 1-ton was used from chapter~\ref{chap:6}, hybrid testing method of the real-scaled building with semi-active controlling MR damper implemented in the chapter~\ref{rthytem} was applied by hybrid testing method technique. This study presents the quantitative evaluation of the seismic performance of a building structure installed with a MR damper using hybrid testing method as described in chapter 2. A building model is identified from the forced vibration testing results of a full-scale five-story building in Chapter 3 and is used as the numerical substructure, and an MR damper corresponding to an experimental substructure is physically tested using a universal testing machine (UTM). First, the force required to drive the displacement of the story, at which the MR damper is located, is measured from the load cell attached to the UTM. The measured force is then returned to a control computer to calculate the response of the numerical substructure. Finally, the experimental substructure is excited by the UTM with the calculated response of the numerical substructure. The hybrid testing method implemented in this study is validated because the real-time hybrid testing results obtained by application of sinusoidal and earthquake excitations and the corresponding analytical results obtained using the Bouc-Wen model as the control force of the MR damper with respect to input currents are in good agreement. Also, the results from hybrid testing method for the passive -on and -off control show that the structural responses did not decrease further by the excessive control force, but decreased due to the increase of the current applied to the MR damper. Also, two semi-active control algorithms (modulated homogeneous friction and the clipped-optimal control algorithms) are applied to the MR damper in order to optimally control the structural responses. To compare the hybrid testing method and numerical results, Bouc-Wen model parameters are identified for each input current. The results of the comparison of experimental and numerical responses show that it is more practical to use hybrid testing method in semi-active devices such as MR dampers. The test results show that a control algorithm can be experimentally applied to the MR damper using hybrid testing method. This study also provides a discussion on each algorithm with respect to the seismic performances.
\\
\\
{\large\textbf{Keywords:}}
real-time hybrid testing method, substructuring technique, tuned liquid type damper, control-structure interaction, inverse transfer function, active tuned mass damper, hybrid mass damper, design of an actuator, 76-story benchmark problem building, force vibration testing, real-scale building, system identification, finite element model updating, magneto-rheological damper, semi-active control algorithms.
\\
\\
Student Number : 72070621\\
\noindent\rule[2pt]{\textwidth}{0.5pt}
