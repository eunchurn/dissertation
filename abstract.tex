\begin{center}
\noindent {\large \textbf{Hybrid Testing Method and Excitation System Design for Seismic and Wind-resistant Performance Evaluation of Building Structures with Nonlinear Dampers}}
\end{center}
\begin{flushright}
Park, Eunchurn\\
Department of Architectural\\
Engineering Graduate School\\
Dankook University\\
\vspace{0.3cm}
Advisor: Professor Min, Kyung-Won
\end{flushright}
{\large\textbf{Abstract:}}
The accurate identification of the dynamic characteristics of a building structure excited by input loads such as an earthquake or wind load is essential not only for the evaluation of the safety and serviceability of the building structures but for the verification of an analytical model used in the seismic or wind design. In the field of system identification (SI) which constructs system matrices describing the accurate input and output relationship, it is critical that input should have enough energy to excite fundamental structural modes and a good quality of output containing structural information should be measured. However, in order to evaluate the seismic performance of real scaled building structures, it is very difficult to build an experimental system and it is almost impossible to perform wind tunnel test real scaled buildings in a laboratory for wind resistant performance evaluation. Therefore, a specimen of a reduced model by using similarity law is fabricated and tested, or a substructure predicted for nonlinear or inelastic behavior is divided and tested as static or pseudo-dynamic. Also, for a real building in use, generally portable excitation system such as an eccentric exciter or a linear mass shakers are installed and actuating the main mode of structures such as harmonic load to evaluate the dynamic performance of the building or to use it as a means for system identification. However, in order to evaluate the accurate seismic and wind performance, it is necessary to excite a building structure so that the characteristics of the seismic load and the characteristics of the wind load are sufficiently reflected in the experiment.

This thesis proposes a hybrid testing method and the controller design of actuators to simulate the responses of a building structure subjected to such dynamic excitations as earthquakes and wind loads. Firstly, The hybrid testing method is a technique to experiment with a non-linear part in an entire structural system as an experimental model and a part that is expected to behave linearly as an analytical model. In this method, the physical unit of the interface between the experimental model and the analytical model should be selected, and an experimental system should be constructed in which the interface of the analytical model is designed to act as the excitation force of the experimental model. In this thesis, a substructure test as part of the hybrid testing method by separating the structural system into the experimental structure and the analytical structure through the shaking table, and the hybrid test of nonlinear vibration control devices, the tuned liquid damper, the tuned liquid column damper and the tuned liquid mass damper as experimental models and building structures as analytical models were designed and implemented to evaluate the performance. Furthermore, seismic performance evaluation of a building with magneto-rheological (MR) damper installed by installing an MR damper, which is a 1-ton inter-story damper with high nonlinearity, on UTM, and the semi-active algorithm applied to the MR damper were experimentally evaluated. This method can provide the possibility and quantitative basis for experimentally finding and applying the optimum algorithm of the control device with unspecific nonlinearity or the performance of the device itself. Secondary, the controller system design that implements the dynamic response is a system that installs a linear mass shaker or an active mass damper on a particular floor and implements a pseudo-seismic or pseudo wind load response in a full-scale building. In this thesis, to simulate of wind-induced responses, a 76-story benchmark building which has an analytical model and wind load data were numerically analyzed, and to realize the seismic responses, the pseudo-earthquake excitation is implemented by the hybrid mass damper with the real-scaled 5-story building. Also, an MR damper with the capacity of 1-ton was installed to evaluate the semi-active control performance of MR damper under earthquakes load in a real scale building. These techniques can be used to assess the seismic and wind resistant performance of real-scaled building structures and to evaluate the seismic and wind resistance performance of buildings with nonlinear control devices.
\\
\\
{\large\textbf{Keywords:}}
hybrid testing method, substructuring technique, tuned liquid type damper, control-structure interaction, inverse transfer function, active tuned mass damper, hybrid mass damper, design of an actuator, 76-story benchmark problem building, force vibration testing, real-scale building, system identification, finite element model updating, magneto-rheological damper, semi-active control algorithms.
\\
\\
Student Number : 72070621\\
\noindent\rule[2pt]{\textwidth}{0.5pt}
