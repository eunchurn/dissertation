%!TEX encoding = UTF-8 Unicode
%\part{Conclusions}
\chapter{Conclusions}

In this thesis, the hybrid testing method (a substructuring technique, TLD-, TLCD- and TLMD-building structure hybrid testing method) for the shaking table test was proposed. The proposed testing technique adopts the upper part of the whole structure or nonlinear control devices as the experimental substructure, which corresponds to a physical test model. The lower part of the entire structure or whole building structure is modeled numerically. In order to verify the validity and accuracy of the proposed technique, a shaking table test was conducted. Also this thesis presents the design of excitation systems for simulating dynamic loads. The controller design of an actuator is presented to simulate the responses of a building structure subjected to such dynamic excitations as earthquakes and wind loads. The result of the study can be summarized as follows.

In chapter~\ref{rthytem}, a hybrid testing method is proposed. First, a substructuring test of a building structure using the hybrid testing method was performed by using a shaking table. Selecting a particular story of a shear-type building structure as an interface, the upper part of the target structure is divided into the experimental substructure, and the lower part is divided into numerical analytic substructure. The boundary load at the interface separated by the substructure is calculated in real time of the absolute acceleration response of the experimental model. The acceleration of the interface between two substructures is calculated by time history analysis of the numerical substructure and used as command signal of the shaking table controller. At this time, the shaking table controller that generates the command signal minimizes the error between the generated command signal and the shaking table motion for the excitation of the experimental substructure. The validity and accuracy of the shaking table substructure test of the hybrid testing method were confirmed by the experimental results and the numerical analysis results. Also, a hybrid testing method for evaluating the control performance of a building structure subjected to earthquake excitation of a tuned liquid type damper (TLD, TLCD, and TLMD), which is a nonlinear control device, was experimentally performed. This experiment is performed by using only control devices (TLD, TLCD, and TLMD), load cell and shaking table without physical building structure because this experiment performs divided into nonlinear control device as an experimental model and building model as an analytical model. The structural responses of the interaction system are calculated numerically in real time using an analytical structure model, a given earthquake record, and a shear force generated by the TLD, TLCD, and TLMD, and the shaking table reproduces both the controlled and uncontrolled absolute acceleration of the TLD, TLCD, and TLMD installed floor by manipulating the feedback gain of the shear force signal measured by the load cell positioned between the TLD/TLCD/TLMD and the shaking table. In this thesis, the results of the hybrid test with those of the existing full-scale structural test are compared, and it is present that accurate results of the seismic performance of buildings with nonlinear vibration control system can be obtained by using hybrid test without a physical model. In this thesis, through not only TLD and TLCD which is verified through previous studies but also TLMD which is newly proposed to reduce bidirectional responses of building structures is evaluated by using a hybrid testing method, a hybrid testing method is beneficial for that the recently proposed control device can be assessed experimentally in the design stage.

In chapter~\ref{chap:6}, a design of an actuator for simulating the wind-induced responses of a building structure is presented. The actuator force is calculated by using the inverse transfer function of a target structural responses to the actuator. Filter and envelope function are used such that the error between wind and actuator induced responses is minimized by preventing the actuator from the exciting, unexpected modal responses and initial transient responses. The analyses result from a 76-story benchmark building problem in which wind load obtained by wind tunnel test is given, indicate that the actuator installed at a particular floor can approximately embody the structural responses induced by the wind load applied to each floor of the structure. The actuator designed by the proposed method can be effectively used for evaluating the wind response characteristics of a building structure in use and for obtaining an accurate analytical model of the building under wind load. Also, a full scale forced vibration test simulating earthquake response is implemented by using a hybrid mass damper. The finite element model of the real-scaled building structure was analytically constructed, and the model was updated using the results experimentally measured by the forced vibration test. Pseudo-earthquake excitation tests showed that the hybrid mass damper induced floor responses coincided with the earthquake induced ones which were numerically calculated based on the updated FE model. Real-scaled MR damper was installed to verify the effectiveness of the pseudo-earthquake experiment in a real scale building. The MR damper-based control systems are realized when an MR damper is designed by deriving a suboptimal design procedure considering optimization problem and magnetic analysis, and then a damper with the capacity of 1.0 ton is manufactured. In the experiments, a linear active mass driver and the linear shaker seismic simulation testing method are used to excite the building structure to match the full-scale building vibrate as if the building undergoes an earthquake. Under the four historical earthquakes and one filtered artificial earthquake, the performance of the semi-active control algorithms including the passive optimal case is experimentally evaluated. From the experimental results, one can conclude that the Lyapunov and semi-active neuro-control algorithms are appropriate in reducing accelerations of the structural system, and the passive optimal case and the maximum energy dissipation algorithm show the excellent performance in reducing the first-floor displacement.

In chapter~\ref{chap:rthytem-mrdamper}, The MR damper with the capacity of 1-ton was used from chapter~\ref{chap:6}, a hybrid testing method of the real-scaled building with semi-active controlling MR damper implemented in the chapter~\ref{rthytem} was applied by hybrid testing method technique. This study presents the quantitative evaluation of the seismic performance of a building structure installed with an MR damper using a hybrid testing method as described in chapter~\ref{rthytem}. A building model is identified from the forced vibration testing results of a full-scale five-story building in chapter~\ref{chap:6} and is used as the numerical substructure, and an MR damper as an experimental substructure is physically tested using a universal testing machine (UTM). First, the force required to drive the displacement of the story, at which the MR damper is located, is measured from the load cell attached to the UTM. The measured force is then returned to a control computer to calculate the response of the numerical substructure. Finally, the experimental substructure is excited by the UTM with the calculated response of the numerical substructure. The hybrid testing method implemented in this thesis is validated because the hybrid testing results obtained by application of sinusoidal and earthquake excitations and the corresponding analytical results obtained using the Bouc-Wen model as the control force of the MR damper with respect to input currents are in good agreement. Also, the results from the hybrid testing method for the passive -on and -off control show that the structural responses did not decrease further by the excessive control force, but decreased due to the increase of the current applied to the MR damper. Also, two semi-active control algorithms (modulated homogeneous friction and the clipped-optimal control algorithms) are applied to the MR damper to optimally control the structural responses. To compare the hybrid testing method and numerical results, Bouc-Wen model parameters are identified for each input current. The results of the comparison of experimental and numerical responses show that it is more practical to use the hybrid testing method in semi-active devices such as MR dampers. The test results indicate that a control algorithm can be experimentally applied to the MR damper using the hybrid testing method. This thesis also provides a discussion on each algorithm concerning the seismic performances.
