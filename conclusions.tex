%!TEX encoding = UTF-8 Unicode
%\part{Conclusions}
\chapter{Conclusions}

In this paper, firstly, a substructuring technique, TLD-, TLCD- and TLMD-building structure hybrid testing method for the shaking table test was proposed. The proposed testing technique adopts the upper part of the whole structure or nonlinear control devices as the experimental substructure, which corresponds to a physical test model. The lower part of the whole structure or whole building structure is modeled numerically. In order to verify the validity and accuracy of the proposed technique, a shaking table test was conducted. The result of the study can be summarized as follows.
\begin{enumerate}
\item To reduce the distortion of the interface acceleration, the inverse transfer function of the shaking table was identified and its state space realization was implemented in the shaking table controller.
\item In this paper, the linear transfer function approach for controlling the motion of a shaking table was considered to experimentally verify the proposed method for a linear experimental part. However, this approach would be inappropriate for a coupled non-linear system leading to experimental instability. Therefore, in such case, the controller using the inverse transfer function of shaking table, shown in Figure~\ref{fig:2-11}, would be modified to compensate an experimental instability.
\item The interface force between the experimental and numerical substructures was obtained using only acceleration measurement and mass information so that high-capacity loads cell and installation jigs are not required in the substructuring technique.
\item The proposed method basing the interface force measurement on acceleration measurements from an experimental substructure is partially available only when the mass distribution is discrete - for example, this technique would be applicable to the TMD as an experimental part. Also, the interface force measurement using force transducers is required to perform the proposed method when wind forces are applied to the experimental substructure.
\item Experimental results demonstrate that the proposed substructuring technique can reproduce the dynamic behavior of the assumed whole structure.
\item An unexpected vibration of the experimental substructure can be induced by the feedback of responses including its inherent natural modes and then by the error occurred in calculating the numerical substructure.
\item It is considered that to minimize the effect of natural modes of an experimental substructure on the substructured system, the structural model as heavily-damped as possible would be used as an experimental part.
\item The proposed technique can be extended to the substructuring technique with the middle part of a whole structure in combination with the conventional substructuring technique employing lower part as the experimental substructure.
\item The TLD installed at the top floor of the structure is physically tested, and simultaneously numerical calculation is carried out for the assumed analytical structural model.
\item Comparison between the structural responses obtained by the hybrid testing method and the conventional shaking table test of a single story steel frame with TLD and TLCD indicates that the performance of the TLD and TLCD can be accurately evaluated using the hybrid testing method without the physical structural model.
\item The uncontrolled and TLD-controlled structural responses of a three-story structure are obtained by the hybrid testing method in both time and frequency domains, showing that TLD can effectively mitigate the seismic responses of building structures and the hybrid testing method can reproduce the dynamic behavior of TLD–structure interaction systems for both the uncontrolled and controlled case.
\item The hybrid testing method can be also applied to the performance evaluation of new designed tuned liquid typed damper which has strong inherent nonlinearity such as the TLMD.
\end{enumerate}

Secondary, this paper presents the design of excitation systems for simulating dynamic loads. The controller design of an actuator is presented to simulate the responses of a building structure subjected to such dynamic excitations as earthquakes and wind loads. The result of the study can be summarized as follows.  

\begin{enumerate}
\item A design of a shaker for simulating wind induced responses of a building structure was presented as a preliminary study for evaluating wind-resistance characteristics of practical building structures.
\item The force of the shaker was obtained using the inverse transfer function of structural responses. Also, band stop filter was used to remove zero of the transfer function such that undesirable modal excitation is prevented, and envelop function was used to reduce the error occurring in transient initial states. 
\item The numerical analyses results from a 76-story benchmark building confirmed that the structural responses of a building structure excited by wind loads acting at all floors could be reproduced by the proposed linear shaker installed at a specific floor.
\item The performances of the excitation systems were dependent on type and position of the target structural response for which acceleration response was suitable because targeting displacement response required large and high-speed changing control force.
\item In order to enhance practical applicability of the wind-response simulating excitation systems, finite element model updating based on measured data, and the scaling or restriction of the excitation force or stroke limit for avoiding damage of the practical building etc. should be considered
\item The field measurement system of full-scale structure and excitation system were established, and then forced vibration test was carried out using the Hybrid Mass Damper designed to simulate seismic load.
\item System identification of full-scale structure was carried out through white noise test and the finite element model is updated from measured data. 
\item The seismic excitation system was accomplished through inverse transfer functions of structure and HMD by the system identifications.
\item The normalized RMS response error through the experimental results showed more increasing one than numerical analysis. This phenomenon is caused by the errors due to phase of the inverse transfer function of HMD caused by compensating for dynamic characteristics of HMD.
\item In simulating the pseudo-earthquake response, unexpected modal responses should be considered, when the excitation signal is generated through the inverse transfer function of structure, because it is measured non-existent errors such as noises of structural response.
\item The effectiveness of the MR damper-based control systems with various control algorithms for seismic protection of full-scale five-story steel frame building structure is experimentally verified in this paper.
\item An MR damper is designed by deriving a suboptimal design procedure considering optimization problem and magnetic analysis, and manufactured into 1.0ton MR damper in order to realize semiactive control systems.
\item Under the four historical earthquakes and one filtered artificial earthquake, various semiactive control algorithms including the passive optimal case are evaluated and compared with one another.
\item The investigation of the hysteretic behavior of an MR damper and the seismic performance evaluation of a building structure, installed with an MR damper, are experimentally implemented using the hybrid testing method.
\item The Bouc-Wen model is used to calculate the control force of the MR damper used in this study, and its parameters are identified based on the experimental results from the hybrid testing method, which used a sinusoidal wave as the ground input acceleration.
\item The hybrid testing method is validated because the hybrid testing results from the sinusoidal and earthquake excitations and the corresponding analytical results agreed well with each other.
\item To compare the results obtained from the hybrid testing method and the numerical analysis, Bouc-Wen model parameters are identified by each input current. The results of this comparison show that the hybrid testing method is more practical than the numerical analysis due to the non-linear variations of the reaction velocity and excitation frequency.
\item The hybrid testing of MR damper results indicated that the seismic performance of a building structure installed with an MR damper can be indirectly evaluated by the hybrid testing method, in which only the MR dampers are physically tested.
\end{enumerate}
