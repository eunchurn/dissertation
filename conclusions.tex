%!TEX encoding = UTF-8 Unicode
%\part{Conclusions}
\chapter{Conclusions}
\section{Concluding Remarks}
This research focused on the design of exciting systems for simulating dynamic loads. The controller design of an actuator is presented to simulate the responses of a building structure subjected to such dynamic excitations as earthquakes and wind loads. Also, the real-time hybrid shaking table testing method (RHSTTM) and the real-time substructuring technique are proposed by using only the physical control device and upper substructure as an experimental part and considering control-structure interaction.
All of testing results shows the good agreement between conventional testing method or numerical analysis and proposed testing method or experimental results.
The result of the study can be summarized as follows.  
(1) The linear transfer function approach for controlling the motion of a shaking table was considered to experimentally verify the proposed method for a linear experimental part. However, this approach would be inappropriate for a coupled non-linear system leading to experimental instability. Therefore, in such case the controller using the inverse transfer function of shaking table would be modified to compensate an experimental instability. 
(2) It is considered that to minimize the effect of natural modes of an experimental substructure on the substructured system, the structural model as havily-damped as possible would be used as an experimental part.
(3) The proposed technique can be extended to the real-time substructuring technique with the middle part of a whole structure in combination with the conventional substructuring technique employing lower part as the experimental substructure. 
(4) Comparison between the structural responses obtained by the RHSTTM and the conventional shaking table test of a single story steel frame with TLD/TLCD indicates that the performance of the TLD/TLCD can be accurately evaluated using the RHSTTM without the physical structural model.
(5) A design of a shaker for simulating wind induced responses of a building structure was presented as a preliminary study for evaluating wind-resistance characteristics of practical building structures. 
(6) The force of the shaker was obtained using the inverse transfer function of structural responses. Also, band stop filter was used to remove zero of the transfer function such that undesirable modal excitation is prevented, and envelop function was used to reduce the error occurring in transient initial states. The numerical analyses results from a 76-story benchmark building confirmed that the structural responses of a building structure excited by wind loads acting at all floors could be reproduced by the proposed linear shaker installed at a specific floor.
(7) The field measurement system of full-scale structure and excitation system were established, and then forced vibration test was carried out using the Hybrid Mass Damper designed to simulate seismic load. 
(8) System identification of full-scale structure was carried out through white noise test and Finite Element (FE) model is updated. The seismic excitation system was accomplished through inverse transfer functions of structure and HMD by the system identifications.
(9) The normalized RMS error through the experimental results showed more increasing one than numerical analysis. This phenomenon is caused by the errors due to each system identification and curve fitting the inverse transfer function of HMD.
(10) In simulating the pseudo-earthquake response, unexpected modal responses should be considered, when the excitation signal is generated through the inverse transfer function of structure, because it is measured non-existent errors such as noises of structural response.

Also, in further study, it needs to consider as follows,
(1) The proposed technique can be extended to the real-time substructuring technique with the middle part of a whole structure in combination with the conventional substructuring technique employing lower part as the experimental substructure.
(2) In order to enhance practical applicability of the wind-response simulating shaker, a FE modeling error, updating of the FE model based on measured data, dynamics of the shaker, and the scaling or restriction of the shaker force limit for avoiding damage of the practical building etc. should be considered in further study.
(3) In order to minimize these excitation system errors of structure, accordingly, It essentially needs to prevent the unexpected modal responses from the inverse transfer functions of HMD and to identify the exact system. Accordingly the band-pass filter or band-stop filter should be designed in order to minimize the unexpected modal responses.


\section{Future Study}